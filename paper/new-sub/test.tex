\documentclass[11pt,a4paper]{article}

\usepackage{graphicx}
\usepackage{amsmath,amssymb}
\usepackage{authblk}
\usepackage{geometry}
\usepackage[numbers]{natbib}
\geometry{margin=1in}

% For S-numbering in Supplement
\usepackage{chngcntr}

% Optional: Keywords command
\newcommand{\keywords}[1]{%
  \vspace{1em}
  \noindent\textbf{Keywords:} #1
}

\title{Main Article Title}

\author[1]{Author One}
\author[2]{Author Two}
\affil[1]{Affiliation One}
\affil[2]{Affiliation Two}

\date{}

\begin{document}

\maketitle

\begin{abstract}
This is the abstract text summarizing the paper.
\end{abstract}

\keywords{keyword1; keyword2; keyword3}

\section{Introduction}
Main text goes here.

\section{Methods}
Methods here.

\section{Results}
Results here.

\section{Conclusion}
Conclusion here.
\cite{neumannNearoptimalFeasibleSpace2021}

% ---------------------
% Supplementary Material
% ---------------------

\clearpage
\section*{Supplementary Material}
\addcontentsline{toc}{section}{Supplementary Material}

% Reset figure and table counters and add 'S' prefix
\counterwithin{figure}{section}
\renewcommand{\thefigure}{S\arabic{figure}}

\counterwithin{table}{section}
\renewcommand{\thetable}{S\arabic{table}}

\setcounter{figure}{0}  % Start figures at S1
\setcounter{table}{0}   % Start tables at S1

\begin{figure}[htbp]
    \centering
    \includegraphics[width=0.6\textwidth]{example-image}  % Replace with your figure
    \caption{This is Supplementary Figure S1.}
    \label{fig:sup1}
\end{figure}

\begin{table}[htbp]
    \centering
    \begin{tabular}{cc}
    \hline
    A & B \\
    \hline
    1 & 2 \\
    3 & 4 \\
    \hline
    \end{tabular}
    \caption{This is Supplementary Table S1.}
    \label{tab:sup1}
\end{table}

% ---------------------
% References
% ---------------------

\clearpage
\bibliographystyle{iopart-num}
\bibliography{references}

\end{document}
