%% 
%% Copyright 2007-2024 Elsevier Ltd
%% 
%% This file is part of the 'Elsarticle Bundle'.
%% ---------------------------------------------
%% 
%% It may be distributed under the conditions of the LaTeX Project Public
%% License, either version 1.3 of this license or (at your option) any
%% later version.  The latest version of this license is in
%%    http://www.latex-project.org/lppl.txt
%% and version 1.3 or later is part of all distributions of LaTeX
%% version 1999/12/01 or later.
%% 
%% The list of all files belonging to the 'Elsarticle Bundle' is
%% given in the file `manifest.txt'.
%% 
%% Template article for Elsevier's document class `elsarticle'
%% with numbered style bibliographic references
%% SP 2008/03/01
%% $Id: elsarticle-template-num.tex 249 2024-04-06 10:51:24Z rishi $
%%
\documentclass[preprint,12pt,sort&compress]{elsarticle}
% \documentclass[final,5p,times,twocolumn,sort&compress]{elsarticle}

%% Custom packages
\usepackage[super]{nth}
\usepackage{amsfonts,amsthm,cancel,siunitx,calculator,calc,mathtools,empheq,latexsym}
\usepackage[version=4]{mhchem}
\usepackage[hyphens]{url} % allow line breaks in urls
\usepackage[breaklinks=true]{hyperref} % allow breakable links
\usepackage{booktabs,multicol,multirow,tabularx,array}
\usepackage[official]{eurosym}
\usepackage{subcaption}
\usepackage{placeins}
\usepackage{float}
\usepackage{tabularx}
\usepackage{booktabs}
\usepackage{enumitem}
\usepackage{bm}

\newcolumntype{R}[1]{>{\raggedleft\arraybackslash}p{#1}}

\graphicspath{
    {figures/}
  }
\DeclareGraphicsExtensions{.pdf,.jpeg,.jpg,.png}

%% Use the option review to obtain double line spacing
%% \documentclass[authoryear,preprint,review,12pt]{elsarticle}

%% Use the options 1p,twocolumn; 3p; 3p,twocolumn; 5p; or 5p,twocolumn
%% for a journal layout:
%% \documentclass[final,1p,times]{elsarticle}
%% \documentclass[final,1p,times,twocolumn]{elsarticle}
%% \documentclass[final,3p,times]{elsarticle}
%% \documentclass[final,3p,times,twocolumn]{elsarticle}
%% \documentclass[final,5p,times]{elsarticle}
%% \documentclass[final,5p,times,twocolumn]{elsarticle}

%% For including figures, graphicx.sty has been loaded in
%% elsarticle.cls. If you prefer to use the old commands
%% please give \usepackage{epsfig}

%% The amssymb package provides various useful mathematical symbols
\usepackage{amssymb}
%% The amsmath package provides various useful equation environments.
\usepackage{amsmath}
%% The amsthm package provides extended theorem environments
%% \usepackage{amsthm}

%% The lineno packages adds line numbers. Start line numbering with
%% \begin{linenumbers}, end it with \end{linenumbers}. Or switch it on
%% for the whole article with \linenumbers.
\usepackage{lineno}
\linenumbers

\journal{Nuclear Physics B}

\begin{document}

\begin{frontmatter}

%% Title, authors and addresses

%% use the tnoteref command within \title for footnotes;
%% use the tnotetext command for theassociated footnote;
%% use the fnref command within \author or \affiliation for footnotes;
%% use the fntext command for theassociated footnote;
%% use the corref command within \author for corresponding author footnotes;
%% use the cortext command for theassociated footnote;
%% use the ead command for the email address,
%% and the form \ead[url] for the home page:
%% \title{Title\tnoteref{label1}}
%% \tnotetext[label1]{}
%% \author{Name\corref{cor1}\fnref{label2}}
%% \ead{email address}
%% \ead[url]{home page}
%% \fntext[label2]{}
%% \cortext[cor1]{}
%% \affiliation{organization={},
%%             addressline={},
%%             city={},
%%             postcode={},
%%             state={},
%%             country={}}
%% \fntext[label3]{}

\title{The role of Projects of Common Interest in reaching Europe's energy policy targets}

%% use optional labels to link authors explicitly to addresses:
%% \author[label1,label2]{}
%% \affiliation[label1]{organization={},
%%             addressline={},
%%             city={},
%%             postcode={},
%%             state={},
%%             country={}}
%%
%% \affiliation[label2]{organization={},
%%             addressline={},
%%             city={},
%%             postcode={},
%%             state={},
%%             country={}}

\author[affi1]{Bobby Xiong\corref{cor1}} %% Author name
\author[affil1]{Iegor Riepin}
\author[affil1]{Tom Brown}

\cortext[cor1]{Corresponding author: \href{mailto:xiong@tu-berlin.de}{xiong@tu-berlin.de}}

%% Author affiliation
\affiliation[affi1]{organization={TU Berlin, Department of Digital Transformation in Energy Systems},
            % addressline={},
            city={Berlin},
            % postcode={},
            % state={},
            country={Germany}}

%% Abstract
\begin{abstract}
  OLD OUTDATED IEW-EXTENDED-ABSTRACT. The European Union (EU) aims to achieve climate-neutrality by 2050, with ambitious 2030 target, such as \SI{55}{\percent} greenhouse gas emissions reduction compared to 1990 levels, \SI{10}{Mt} p.a. of a domestic green \ce{H2} production, and \SI{50}{Mt} p.a. of \ce{CO2} injection capacity, which should be sequestered within the EU. 
  The European Commission selects so-called Projects of Common Interest (PCI) and Projects of Mutual Interest (PMI)---a large infrastructure projects for electricity, hydrogen and \ce{CO2} transport, and storage---that are of transnational importance as they link the energy systems of European countries.
  In this work, we evaluate the impact of PCI-PMI projects for the European energy system and EU energy policies. To achieve this, we investigate how delays in these projects could affect the EU's policy targets and examine potential conflicts between various policy objectives and the overarching greenhouse gas reduction goal. 
  Our preliminary results for 2030 indicate that policy targets can be met even without PCI-PMI projects; however, these projects offer additional benefits: (i) \ce{H2} pipelines enhance the affordability and distribution of green \ce{H2}, thereby jumpstarting the hydrogen economy, and (ii) \ce{CO2} transport projects connect major industrial emissions to offshore sequestration sites in the North Sea.
  In our future work, we will analyse long-term pathway effects up to 2050, and incorporate hybrid interconnectors and \ce{CO2} shipping routes from the PCI-PMI list. 
  Overall, our findings highlight the critical interplay between cross-border cooperation, infrastructure investments, and policy targets in the European energy transition across all sectors.
\end{abstract}

% %%Graphical abstract
% \begin{graphicalabstract}
% %\includegraphics{grabs}
% \end{graphicalabstract}

% %%Research highlights
% \begin{highlights}
% \item Research highlight 1
% \item Research highlight 2
% \end{highlights}

%% Keywords
\begin{keyword}
energy system modelling \sep policy targets \sep infrastructure \sep resilience \sep hydrogen \sep carbon \sep Europe 
%% keywords here, in the form: keyword \sep keyword

%% PACS codes here, in the form: \PACS code \sep code

%% MSC codes here, in the form: \MSC code \sep code
%% or \MSC[2008] code \sep code (2000 is the default)

\end{keyword}

\end{frontmatter}

%% Add \usepackage{lineno} before \begin{document} and uncomment 
%% following line to enable line numbers
%% \linenumbers

%% main text
%%

%% Use \section commands to start a section.
\section*{List of abbreviations}

\begin{itemize}[left=0pt, label={}, itemsep=0pt, parsep=0pt, topsep=0pt]
  \item \textbf{AC} \enspace Alternating Current
  \item \textbf{API} \enspace Application Programming Interface
  \item \textbf{CC} \enspace Carbon Capture
  \item \textbf{CU} \enspace Carbon Utilisation
  \item \textbf{CS} \enspace Carbon Storage
  \item \textbf{CCUS} \enspace Carbon Capture, Utilisation, and Storage
  \item \textbf{DAC} \enspace Direct Air Capture
  \item \textbf{DC} \enspace Direct Current
  \item \textbf{EU} \enspace European Union
  \item \textbf{GHG} \enspace Greenhouse gas
  \item \textbf{NEP} \enspace Netzentwicklungsplan (German grid development plan)
  \item \textbf{NUTS} \enspace Nomenclature of Territorial Units for Statistics
  \item \textbf{PCI} \enspace Projects of Common Interest
  \item \textbf{PMI} \enspace Projects of Mutual Interest
  \item \textbf{REST} \enspace Representational State Transfer 
  \item \textbf{tsam} \enspace Time Series Aggregation Module
  \item \textbf{TYNDP} \enspace Ten-Year Network Development Plan
  \item \textbf{WACC} \enspace Weighted Average Cost of Capital

\end{itemize}

\section{Introduction}
\label{sec:introduction}
WORK-IN-PROGRESS-INCOMPLETE. On the pathway to a climate-neutral Europe by 2050, the European Union (EU) has set ambitious targets for 2030. These targets include a reduction of \SI{55}{\percent} in greenhouse gas emissions compared to 1990 levels \cite{europeancommissionFit55Delivering2021}, \SI{10}{Mt} p.a. domestic green \ce{H2} production \cite{europeancommissionREPowerEUPlanCommunication2022}, and \SI{50}{Mt} p.a. of \ce{CO2} injection capacity with sequestration in within the EU \cite{europeanparliamentRegulationEU20242024}.

To support reaching these targets, the European Commission bi-annually identifies a list of Projects of Common Interest (PCI), which are key cross-border infrastructure projects that link the energy systems of the EU members, including transmission and storage projects for electricity, hydrogen and \ce{CO2} \cite{europeancommissionCommissionDelegatedRegulation2023}. 
The pool of project sutable for PCI status is based on projects submitted by transmission system operators, consortia, or third parties. Projects of Mutual Interest (PMI) further include cooperations with countries outside the EU, such as Norway or the United Kingdom. With a PCI-PMI status, project awardees receive strong political support and are, amongst others, eligible for financial support (e.g. through funding of the Connecting Europe Facility) and see accelerated permitting processes. On the other hand, project promoters are obliged to undergo comprehensive reporting and monitoring processes. 
In order for projects to be eligible for PCI-PMI status, their \textit{potential benefits need to outweigh their costs} \cite{europeancommissionCommissionDelegatedRegulation2023}. Given the political and lighthouse character, these projects are highly likely to be implemented. However, any large infrastructure project, including PCI-PMI projects, commonly face delays due to permitting, financing, procurement bottlenecks, etc. \cite
{acerConsolidatedReportProgress2023}.

\begin{itemize}
  \item Net zero law by 2050 \citet{europeanparliamentRegulationEU20242024}
\end{itemize}

\subsection{Fuels, carriers, targets}
\paragraph{Hydrogen (\ce{H2})}
\begin{itemize}
  \item "net zero systems: H2 feedstock for synthetic fuels, fuel transportation sector, feedstock and heat source in industry," \cite{greevenbroekLittleLoseCase2024}, \cite{beresWillHydrogenSynthetic2024}
\end{itemize}

\subsection{Projects of Common/Mutual Interest}


\newpage
\section{Literature review}
\label{sec:literature_review}
We structure the literature review into three main sections: (i) the value of \ce{CO2} and \ce{H2} in low-carbon energy systems, (ii) transporting \ce{CO2} and \ce{H2} through pipelines, and (iii) addressing uncertainty in energy system models. Based on this review, identify research gaps and position our work as a novel contribution to the current state of the art (iv).

\subsection{The value of \ce{CO2} and \ce{H2} in low-carbon energy systems}
A growing body of literature has been investigating the long-term role of \ce{H2} and \ce{CO2} in low-carbon or net-zero energy systems. Both carriers see their primary value outside the electricity sector, i.e., in the decarbonisation of hard-to-abate sectors such as industry, transport, shipping, and aviation \cite{reigstadMovingLowcarbonHydrogen2022}. While there are direct use cases for \ce{H2} in the industry sector such as steel production, it is primarily expected to serve as a precursor for synthetic fuels, including methanol, Fischer-Tropsch fuels (e.g. synthetic kerosene and naphta) and methane. The demand for these fuels is driven by the aviation, shipping, industry, and agriculture sectors \cite{neumannPotentialRoleHydrogen2023}. To produce these carbonaceous fuels, \ce{CO2} is required as a feedstock (Carbon Utilisation --- CU). This \ce{CO2} can be captured from the atmosphere via Direct Air Capture (DAC) or from industrial and process emissions (e.g. cement, steel, ammonia production) in combination with Carbon Capture (CC) units.

Van Greevenbroek et al. \cite{greevenbroekLittleLoseCase2024}: Look at near optimal solution space by assessing a wide range. Derived from a wide set of literature, modelling hydrogen and CS, CU \cite{fleiterHydrogenInfrastructureFuture2025,beresWillHydrogenSynthetic2024,blancoPotentialHydrogenPowertoLiquid2018,pickeringDiversityOptionsEliminate2022,schreyerDistinctRolesDirect2024,seckHydrogenDecarbonizationEnergy2022,neumannPotentialRoleHydrogen2023,zeyenEndogenousLearningGreen2023,kountourisUnifiedEuropeanHydrogen2024}

Range of assessed CO2 sequestration potential from 275 Mt p.a., 550 Mt p.a., up to 1100 Mt p.a. Range of green hydrogen production in 2050 goes up to 90 Mt p.a. Page 3: Europe has little to loose by committing to targets like 25 Mt pa H2 production by 2040, moderate target, feasible.

"Cost optimal modelling results with a central planning approach may not capture system designs that are politically more viable but slightly more costly." from Koens paper, \cite{trutnevyteDoesCostOptimization2016}

\subsection{Transporting \ce{H2} and \ce{CO2} through pipelines}
Recent publications show that transporting \ce{CO2} and \ce{H2} via dedicated pipeline infrastructure can unlock additional benefits and net cost-savings in a sector-coupled energy system. Victoria et al. \cite{victoriaSpeedTechnologicalTransformations2022a} ... TODO

Neumann et al. \cite{neumannPotentialRoleHydrogen2023} examine the interaction between electricity grid expansion and a European-wide deployment of hydrogen pipelines in a net-zero system (new and retrofitting of existing gas pipelines). While \ce{H2} pipelines are not essential, their build-out can significantly reduce system costs by up to 26 bn. \euro{} p.a. (\SI{3.4}{\percent} of annual CAPEX and OPEX) by connecting regions with excessive renewable potential to storage sites and load centres. 
Extending their previous work, Neumann et al. \cite{neumannEnergyImportsInfrastructure2024} investigate the trade-off between relying on different energy import strategies and domestric infrastructure build-out. By coupling the global energy supply chain model TRACE \cite{hamppImportOptionsChemical2023} and the sector-coupled PyPSA-Eur model, they assess different energy vector import combinations (e.g. electricity, \ce{H2} or \ce{H2} derivatives) and their impact on Europe's infrastructural needs. Depending on the import costs, they observe up to \SI{14}{\percent} in system cost savings. Further, with an increasing share of \ce{H2} imports, the need for domestic \ce{H2} pipelines would decrease. 

In a study by Kontouris et al. \cite{kountourisUnifiedEuropeanHydrogen2024}, the authors explore pathways for a potential integrated hydrogen infrastructure in Europe while considering sector-coupling and energy imports. Using the European energy system model Balmorel \cite{wieseBalmorelOpenSource2018}, the authors implement three scenarios varying between domestic and imported \ce{H2} levels as well as \ce{H2} production technologies. In their findings they identify main \ce{H2} transport corridors from Spain and France, Ireland and the United Kingdom, Italy, and Southeastern Europe. When synergies through sector-coupling are exploited, domestic \ce{H2} production can be competitive, seeing an increase in up to \SI{3}{\percent} in system costs.

Fleiter et al. \cite{fleiterHydrogenInfrastructureFuture2025} use a mixed simulation and optimisation method to model \ce{H2} uptake and transport by coupling three models, (i) FORECAST for buildings and industry, (ii) ALADIN for transport together with (iii) the European energy system model Enertile. Total demand for \ce{H2} ranges from \SI{690}{TWh} to \SI{2800}{TWh} in 2050, \SI{600}{TWh} to \SI{1400}{TWh} for synthetic fuels. In their study, the  chemical and steel industry in Northwest Europe (incl. western regions of Germany, Netherlands and northern regions of Belgium), display a demand of more than \SI{100}{TWh} each. With regard to crossborder transport, they mainly obser hydrogen flows from Norway, UK and Ireland to continental Europe (around \SI{53}{TWh} to \SI{72}{TWh}). Depending on the scenario, the Iberian Peninsula exports around \SI{72}{TWh} to \SI{235}{TWh} via and to France.

On the carbon networks side, \cite{bakkenLinearModelsOptimization2008}

Doing both:
Hofmann et al. \cite{hofmannH2CO2Network2025} address previous research gap in assessing the interaction between \ce{H2} and \ce{CO2} infrastructure, including their production, transport, storage, utilisation, and sequestration. They find that ...
WORK-IN-PROGRESS-INCOMPLETE. 
\subsection{Addressing uncertainty in energy system models}
WORK-IN-PROGRESS-INCOMPLETE. 

\begin{itemize}
  \item Regret analysis common in economics, also in energy system modelling
  \item Carbon networks
  \item Regret
  \item Cite Hobbs, Iegor, Möbius and Riepin two-stage, stochastic, regret approach  \cite{mobiusRegretAnalysisInvestment2020a}
  PCI projects gas
  
\end{itemize}


\section{Research gaps and our contribution}

TODO NOVELTIES:
\begin{itemize}
  \item basically mega PINT CBA, which was not done before, neither for PCI projects nor for the sectors
  \item Chicken and egg problem. Assess real planned projects
  \item high spatial and temporal resolution
  \item regret matrix approach
  \item Time, myopic, iterative dimension, usually studies look directly at the target 2050, yielding overly optimistic results (overnight 2050 optimisation will yield different result than pathway-dependent solutions)
\end{itemize}

This paper aims to evaluate the impact of PCI-PMI projects on the European energy system and EU energy policies. We focus on the following key research questions:

\begin{enumerate} 
  \item What is the impact of delay in PCI-PMI projects' realisation on the EU's policy targets for 2030?
  \item What are the costs associated with adhering to the EU policy targets, even if PCI-PMI projects are delayed? 
  \item Do the green hydrogen production and carbon sequestration targets conflict with the cost-effective achievement of the greenhouse gas emission reduction goals? 
\end{enumerate}

Key motivations for the questions as the EU targets especially for 2030 have have been criticised as unrealistic, primarily politically motivated. \cite{europeancourtofauditorsEUsIndustrialPolicy2024,greevenbroekLittleLoseCase2024}


\newpage
\section{Methodology}
\label{sec:methodology}

We build on the open-source, sector-coupled energy system model PyPSA-Eur \cite{neumannPotentialRoleHydrogen2023,frysztackiComparisonClusteringMethods2022,glaumOffshorePowerHydrogen2024,horschPyPSAEurOpenOptimisation2018} to optimise investment and dispatch decisions in the European energy system. The model's endogenous decisions include the expansion and dispatch of renewable energy sources, dispatchable power plants, electricity storage, power-to-X conversion capacities, and transmission infrastructure for power, hydrogen, and \ce{CO2}. It also encompasses heating technologies and various hydrogen production methods (gray, blue, green).
PyPSA-Eur integrates multiple energy carriers (e.g., electricity, heat, hydrogen, \ce{CO2}, methane, methanol, liquid hydrocarbons, and biomass) with corresponding conversion technologies across multiple sectors (i.e., electricity, transport, heating, biomass, industry, shipping, aviation, agriculture and fossil fuel feedstock). The model features high spatial and temporal resolution across Europe, incorporating existing power plant stocks \cite{gotzensPerformingEnergyModelling2019}, renewable potentials, and availability time series \cite{hofmannAtliteLightweightPython2021}. It includes the current high-voltage transmission grid (AC \SI{220}{kV} to \SI{750}{kV} and DC \SI{150}{kV} and above) \cite{xiongModellingHighvoltageGrid2025}. Furthermore, electricity transmission projects from the TYNDP (SOURCE) and German Netzentwicklungsplan (SOURCE) are also enabled.

\subsection{Model setup}
\label{sec:model_setup}


\paragraph{Temporal resolution}
\label{sec:temporal_resolution}
To assess the long-term impact of PCI-PMI projects on European policy targets across all sectors, we optimise the sector-coupled network for three key planning horizons 2030, 2040, and 2050, myopically. The myopic approach ensures that investment decisions across all planning horizons are coherent and build on top of the previous planning horizon. We use the built-in Time Series Aggregation Module (tsam) to solve the model for 2190 time steps, yielding an average resolution of four hours. tsam is a Python package developed by Kotzur et al. \cite{kotzurImpactDifferentTime2018} to aggregate time series data into representative time slices to reduce computational complexity while maintaining their specific intertemporal characteristics, such as renewable infeed variability, demand fluctuations, and seasonal storage needs.

\paragraph{Geographical scope} 
\label{sec:geographical_scope}
We model 34 European countries, including 25 of the EU27 member states (excluding Cyprus and Malta), as well as Norway, Switzerland, the United Kingdom, Albania, Bosnia and Herzegovina, Montenegro, North Macedonia, Serbia, and Kosovo. Regional clustering is based on administrative NUTS boundaries, with higher spatial resolution applied to regions hosting planned PCI-PMI infrastructure, producing 99 onshore regions (see Table \ref{tab:regional_clustering}). Depending on the scenario, additional offshore buses are introduced to appropriately represent offshore sequestration sites and PCI-PMI projects. To isolate the effect of PCI-PMI projects, Europe is self-sufficient in our study, i.e., we do not allow any imports or exports of the assessed carriers like electricity, \ce{H2}, or \ce{CO2}. 

\paragraph{Technology assumptions} 
\label{sec:technology_assumptions}
As part of the PyPSA-Eur model, we source all technology-specific assumptions including lifetime, efficiency, investment and operational costs from the public \textit{Energy System Technology Data} repository, v.0.10.1 \cite{zeyenPyPSATechnologydataV01012025}. We use values projected for 2030 and apply a discount rate of \SI{7}{\percent}, reflecting the weighted average cost of capital (WACC). We assume \ce{CO2} sequestration costs of 15 \euro{}/t\ce{CO2} which can be considered in the mid-range of the cost spectrum (cf. TODO SOURCE 1 and 10 \euro{}/t\ce{CO2} \cite{hofmannH2CO2Network2025}) 

\paragraph{Demand and \ce{CO2} emissions}
\label{sec:demand_and_co2_emissions}
Energy and fuel carrier demand in the modelled sectors, as well as non-abatable \ce{CO2} process emissions are taken from various sources \cite{mantzosJRCIDEES20152018,eurostatCompleteEnergyBalances2022,manzGeoreferencedIndustrialSites2018,muehlenpfordtTimeSeries2019,krienOemofDemandlibV0222025} and are shown in Figure \ref{fig:exogenous_demand}. Regionally and temporally resolved demand includes electricity, heat, gas, biomass and transport. Internal combustion engine vehicles in land transport are expected to fully phase out in favour of electric vehicles by 2050 \cite{zeyenShiftingBurdensHow2025a}. Demand for hydrocarbons, including methanol and kerosene are primarily driven by the shipping, aviation and industry sector and are not spatially resolved.
To reach net-zero \ce{CO2} emissions by 2050, the yearly emission budget follows the EU's 2030 (\SI{-55}{\percent}) and 2040 (\SI{-90}{\percent}) targets \cite{europeancommissionFit55Delivering2021, europeancommission.directorategeneralforclimateaction.IndepthReportResults2024}, translating into a carbon budget of 2072 Mt p.a. in 2030 and 460 Mt p.a. in 2040, respectively (see Table \ref{tab:targets}).

\paragraph{PCI-PMI projects implementation}
\label{sec:pci-pmi_projects_implementation}

We implement all PCI-PMI projects of the electricity, \ce{CO2} and \ce{H2} sectors (excl. offshore energy islands and hybrid interconnectors, as they are not the focus of our research) by accessing the REST API of the PCI-PMI Transparency Platform and associated public project sheets provided by the European Commission \cite{europeancommissionPCIPMITransparencyPlatform2024}. We add all \ce{CO2} sequestration sites and connected pipelines, \ce{H2} pipelines and storage sites, as well as proposed pumped-hydro storages and transmission lines (AC and DC) to the PyPSA-Eur model. We consider the exact geographic information, build year, as well as available static technical parameters when adding individual assets to the respective modelling year. An overview of the implemented PCI-PMI projects is provided in Figure \ref{fig:regional_scope_map}.
\begin{figure}[htbp]
  \centering
  \includegraphics[width=0.8\linewidth]{map_adm_pcipmi}
  \caption{Map of the regional scope including clustered onshore (grey) and offshore regions (blue), as well as PCI-PMI \ce{CO2} and \ce{H2} pipelines, storage and sequestration sites. Depleted offshore oil and gas fields (red) provide additional \ce{CO2} sequestration potential \cite{hofmannH2CO2Network2025}.}
  \label{fig:regional_scope_map}
\end{figure}

Our implementation can adapt to the needs and configuration of the model, including selected technologies, geographical and temporal resolution, as well as the level of sector-coupling. Here, all projects are mapped to the 99 NUTS regions, in this process, pipelines are are aggregated and connect all overpassing regions. Similar to how all electricity lines and carrier links are modelled in PyPSA-Eur, lengths are calculated using the haversine formula multiplied by a factor of 1.25 to account for the non-straight shape of pipelines.
We apply standardised cost assumptions \cite{zeyenPyPSATechnologydataV01012025} across all existing brownfield assets, model-endogenously selected projects, and exogenously specified PCI-PMI projects, equally. Our approach is motivated by two key considerations: (i) cost data submitted by project promoters are often incomplete and may differ in terms of included components, underlying assumptions, and risk margins; and (ii) applying uniform cost assumptions ensures comparability and a level playing field across all potential investments, including both PCI-PMI and model-endogenous options. 

\paragraph{\ce{CO2} sequestration sites}
\label{sec:co2_sequestration_sites}
Beyond \ce{CO2} sequestration site projects included in the latest PCI-PMI list (around 114 Mt p.a.), we consider additional technical potential from the European \ce{CO2} storage database \cite{europeancommissionEuropeanCO2Storage2020,hofmannH2CO2Network2025}. While social and commercial acceptance of \ce{CO2} storage has been increasing in recent years, however, concerns still exist regarding its long-term purpose and safety \cite{vanalphenSocietalAcceptanceCarbon2007}. For this reason, we only consider conservative estimates from depleted oil and gas fields, which are primarily located offshore in the British, Norwegian, and Dutch North Sea (see Figure \ref{fig:regional_scope_map}), yielding a total sequestration potential of \SI{7164}{Mt}. Spread over a lifetime of 25 years, this translates into an annual sequestration potential of up to 286 Mt p.a. We then cluster all offshore potential within a buffer radius of \SI{50}{km} per offshore bus region in each modelled NUTS region and connect them through offshore \ce{CO2} pipelines to the closest onshore bus (TODO: add reference to cost assumptions in appendix). 


\subsection{Scenario setup and regret matrix}
\label{sec:scenario_setup}
To assess the long-term impact of PCI-PMI projects on the European energy system and EU energy policies, we implement a regret-matrix based approach. This allows us to evaluate the performance of a set of long-term scenarios under three different short-term occurrences for each planning horizon, individually (Table \ref{tab:regret_matrix_setup}).
\subsubsection{Long-term scenarios}
\paragraph{Scenario definition}
\label{sec:definition}
We define the long-term scenarios based on the degree of \ce{CO2} and \ce{H2} infrastructure build-out, including the roll-out of PCI-PMI projects as well additional pipeline investments. In total, we implement five long-term scenarios, (i) a pessimistic scenario (Decentral Islands --- DI) without any \ce{H2} pipeline and onshore \ce{CO2} pipeline infrastructure, (ii) a scenario that considers the on-time commissioning of all PCI-PMI \ce{CO2} and \ce{H2} projects (PCI-PMI --- PCI) only, (iii) more ambitious scenarios that further allow investments into national and (iv) international pipelines (PCI-PMI nat. --- PCI-n and PCI-PMI internat. --- PCI-in), and (v) a scenario that does not assume any fixed PCI-PMI infrastructure but allows for a centralised, purely needs-based build-out of \ce{CO2} and \ce{H2} pipelines (Centralised Planning --- CP). An overview of the long-term scenarios and their associated model-endogenous decision variables is provided in Table \ref{tab:long-term_scenarios}. 

\begin{table*}[htbp]
  \centering
  \caption{Overview of long-term scenarios and their key assumptions.}
  \label{tab:long-term_scenarios}
  \scriptsize
  \begin{tabularx}{\textwidth}{R{3.9cm}>{\centering\arraybackslash}X>{\centering\arraybackslash}X>{\centering\arraybackslash}X>{\centering\arraybackslash}X>{\centering\arraybackslash}X}
    \toprule
    \textbf{Long-term scenarios} & 
    \textbf{DI} & 
    \textbf{PCI} & 
    \textbf{PCI-n} & 
    \textbf{PCI-in} & 
    \textbf{CP} \\
    \midrule
    \textbf{\ce{CO2} sequestration} & & & & & \\
    Depleted oil \& gas fields* & $\blacksquare$ & $\blacksquare$ & $\blacksquare$ & $\blacksquare$ & $\blacksquare$ \\
    PCI-PMI seq. sites** & -- & $\blacksquare$ & $\blacksquare$ & $\blacksquare$ & $\blacksquare$ \\
    \midrule
    \textbf{\ce{H2} storage} & & & & & \\
    Endogenous build-out & $\blacksquare$ & $\blacksquare$ & $\blacksquare$ & $\blacksquare$ & $\blacksquare$ \\
    PCI-PMI storage sites & -- & $\blacksquare$ & $\blacksquare$ & $\blacksquare$ & $\blacksquare$ \\
    \midrule
    \textbf{\ce{CO2} pipelines} & & & & & \\
    to depleted oil \& gas fields & $\blacksquare$ & $\blacksquare$ & $\blacksquare$ & $\blacksquare$ & $\blacksquare$ \\
    to PCI-PMI seq. sites & -- & $\blacksquare$ & $\blacksquare$ & $\blacksquare$ & $\blacksquare$ \\
    \midrule
    \textbf{\ce{CO2} and \ce{H2} pipelines} & & & & & \\
    PCI-PMI & -- & $\blacksquare$ & $\blacksquare$ & $\blacksquare$ & $\blacksquare$ \\
    National build-out & -- & $\blacksquare$ & $\blacksquare$ & $\blacksquare$ & $\blacksquare$ \\
    International build-out & -- & -- & -- & $\blacksquare$ & $\blacksquare$ \\
    PCI-PMI extendable & -- & -- & -- & -- & $\blacksquare$ \\

    \bottomrule
  \end{tabularx}
  \caption*{\scriptsize $\blacksquare$ enabled \quad -- disabled \quad * approx. 286 Mt p.a. \quad ** approx. 114 Mt p.a.}
\end{table*}

\paragraph{Targets}
\label{sec:targets}
In all long-term scenarios, emission, technology, sequestration and production targets have to be met for each planning horizon (see Table \ref{tab:targets}). For the year 2030, these targets are directly derived from the EU's policy targets, including a \SI{55}{\percent} reduction in greenhouse gas emissions compared to 1990 levels \cite{europeancommissionFit55Delivering2021}, \SI{10}{Mt} p.a. of domestic green \ce{H2} production \cite{europeancommissionREPowerEUPlanCommunication2022} and \SI{40}{GW} \cite{europeancommissionCommunicationCommissionEuropean2020}, and \SI{50}{Mt} p.a. of \ce{CO2} sequestration capacity \cite{europeanparliamentRegulationEU20242024}. For 2050, the \ce{CO2} sequestration target is derived from impact assessment, modelling for European Commission's 2024 industrial carbon management strategy, in which 250 Mt p.a. out of 450 Mt p.a. (Carbon Capture Utilisation and Storage) is sequestered \cite{europeancommissionCommunicationCommissionEuropean2024}. \ce{H2} production targets for 2050 are based on the European Commission's METIS 3 study S5 \cite{europeancommission.directorategeneralforenergy.METIS3Study2023}, modelling possible pathways for industry decarbonisations until 2040. For 2040, we interpolate linearly between the 2030 and 2050 targets. The electrolyser capacities for 2040 and 2050 are scaled by the ratio of \ce{H2} production to electrolyser capacity in 2030. An overview of the targets and their values is provided in Table \ref{tab:targets}. Note that we implement the green \ce{H2 production} target as a minimum \ce{H2} production constraint from elecrolysers, hence we will refer to this \ce{H2} as electrolytic \ce{H2} within the scope of this paper.

\begin{table*}[htbp]
  \centering
  \caption{Pathway for implemented targets.}
  \label{tab:targets}
  \scriptsize
  \begin{tabularx}{\textwidth}{R{3.9cm}>{\centering\arraybackslash}X>{\centering\arraybackslash}X>{\centering\arraybackslash}X}
    \toprule
    \textbf{Planning horizon} & \textbf{2030} & \textbf{2040} & \textbf{2050} \\
    \midrule
    \textbf{Targets} & & & \\
    GHG emission reduction &  \SI{-55}{\percent} & \SI{-90}{\percent} & \SI{-100}{\percent} \\
    \ce{CO2} sequestration & 50 Mt p.a. & 150 Mt p.a. & 250 Mt p.a. \\
    Electrolytic \ce{H2} production & 10 Mt p.a. & 27.5 Mt p.a. & 45 Mt p.a. \\
    \ce{H2} electrolyser capacity & \SI{40}{GW} &  \SI{110}{GW} &  \SI{180}{GW} \\
    \bottomrule
  \end{tabularx}
  \caption*{\scriptsize Model targets based on \cite{europeancommissionFit55Delivering2021,europeancommissionREPowerEUPlanCommunication2022,europeanparliamentRegulationEU20242024,europeancommissionCommunicationCommissionEuropean2024,europeancommission.directorategeneralforenergy.METIS3Study2023}}
\end{table*}

\subsection{Short-term scenarios}
\label{sec:short-term_scenarios}

In a second step, we assess the impact of three short-term scenarios on the long-term scenarios, by fixing or removing pipeline capacities (depending on the scenario). Further, the model can still react by investing into additional generation, storage, or conversion, or carbon-removal technologies in the short-term, assuming the technical potential was not exceeded in the long-term optimisation. In \textit{Reduced targets}, we remove all of the long-term targets (Table \ref{tab:targets}) except for the GHG emission reduction targets to assess the value of the \ce{CO2} and \ce{H2} infrastructure in a less ambitious policy environment. In \textit{Delayed pipelines}, we assume that all PCI-PMI and endogenous pipelines are delayed by one period, i.e., the commissioning of the project is shifted to the next planning horizon. Lastly, we remove all pipeline capacities in \textit{No pipelines}, including the PCI-PMI projects, allowing us to evaluate the impact of a complete lack of planned infrastructure. 

Table \ref{tab:regret_matrix_setup} gives an overview of this regret-analysis and their individual assumptions, where the long-term scenario serves as the \textit{planned} or \textit{anticipated} and the short-term scenario serves as the hypothetically \textit{realised} outcome. By comparing the system costs of related long-term and short-term scenarios, we can calculate its associated economic regret. 
In total, we run 60 optimisations on a cluster, taking up to 160 GB of RAM and 8 to 16 hours each to solve: ($n_{LT} \times n_{planning\,horizons}) \times (1+n_{ST}) = 60$. The models are solved using Gurobi.

\begin{table*}[htbp]
  \centering
  \caption{Regret matrix setup: Long-term and short-term scenarios.}
  \label{tab:regret_matrix_setup}
  \scriptsize
  \begin{tabularx}{\textwidth}{R{3.9cm}>{\centering\arraybackslash}X>{\centering\arraybackslash}X>{\centering\arraybackslash}X}
    \toprule
    \textbf{Short-term} & \textbf{Reduced targets} & \textbf{Delayed pipelines} & \textbf{No pipelines} \\
    \midrule
    \textbf{Long-term scenarios} & & & \\
    Decentral Islands (\textbf{DI}) & $\blacksquare$ & -- & -- \\
    PCI-PMI (\textbf{PCI}) & $\blacksquare$ & $\blacksquare$ & $\blacksquare$ \\
    PCI-PMI nat. (\textbf{PCI-n}) & $\blacksquare$ & $\blacksquare$ & $\blacksquare$\\
    PCI-PMI internat. (\textbf{PCI-in}) & $\blacksquare$ & $\blacksquare$ & $\blacksquare$ \\
    Central Planning (\textbf{CP}) & $\blacksquare$ & $\blacksquare$ & $\blacksquare$ \\
    \midrule
    \textbf{Targets} & & & \\
    GHG emission reduction &  $\blacksquare$ &  $\blacksquare$ &  $\blacksquare$ \\
    \ce{CO2} sequestration &  -- &  $\blacksquare$ &  $\blacksquare$ \\
    Electrolytic \ce{H2} production &  -- &  $\blacksquare$ &  $\blacksquare$ \\
    \ce{H2} electrolysers &  -- &  $\blacksquare$ &  $\blacksquare$ \\
    \midrule
    \textbf{\ce{CO2} + \ce{H2} infrastructure} & & & \\
    \ce{CO2} sequestration sites & $\blacksquare$ &  $\blacksquare$ &  $\blacksquare$ \\
    \ce{CO2} pipelines to seq. site & $\blacksquare$ &  $\blacksquare$ &  $\blacksquare$ \\
    \ce{CO2} pipelines & $\blacksquare$ &  $\square$ &  -- \\
    \ce{H2} pipelines & $\blacksquare$ &  $\square$ &  -- \\
    \bottomrule
  \end{tabularx}
  \caption*{\scriptsize $\blacksquare$ enabled \quad $\square$ delayed by one period \quad -- disabled}
\end{table*}

\section{Results and discussion}
\label{sec:results_and_discussion}
We structure the results and discussion into three main sections. First, we present the results of the long-term scenarios. Then, we look at the impact of the short-term scenarios on the long-term scenarios, by comparing the economic regret and impacts on \ce{CO2} and \ce{H2 balances}. Finally, we assess the benefits of the PCI-PMI projects with regard to reduced system costs and discuss the implications of our findings for the European energy system and its policy targets. 

\subsection{Long-term scenarios}
\label{sec:long-term_scenarios}
In all long-term runs, we observe the highest total annual system costs in the planning horizon 2040, ranging from 912 to 968 bn. \euro{} p.a. (Figure \ref{fig:costs_overview}), driven by high investments. This can be primarily attributed to the strict exogenously given GHG emission reduction pathway, facing the largest net change from 2030 to 2040 --- a carbon budget reduction of more than 1600 Mt p.a. as opposed to the remaining 460 Mt p.a. in the last decade. In 2030, total system costs are lowest in the \textit{DI} and \textit{CP} scenario, as the model does not see the need for large-scale investments into \ce{H2} and \ce{CO2} infrastructure yet. With \ce{CO2} pipelines connecting depleted offshore oil and gas fields to their closest onshore region, the policy targets, incl. \ce{CO2} sequestration can be achieved at a total of 865 bn. \euro{} p.a. Adding PCI-PMI projects in 2030 increases costs by less than \SI{1}{\percent}. 

\begin{figure}[htbp]
  \centering
  \includegraphics[width=\linewidth]{costs_overview.pdf}
  \caption{Total annual system costs (CAPEX + OPEX) by technology group.}
  \label{fig:costs_overview}
\end{figure}

Starting in 2040, all scenarios with PCI-PMI and endogenous pipeline investments unlock significant cost savings, from more than 30 bn. \euro{} p.a. in the \textit{PCI} up to 50 bn. \euro{} p.a. in the \textit{PCI-in} scenario. By giving the model complete freedom in pipeline expansions, additional annual cost savings of 6 to 7 bn. \euro{} are unlocked by investing in fewer, but more optimally located \ce{CO2} and \ce{H2} pipelines from a systemic perspective (see \textit{PCI-in} pipeline utilisation in Figures \ref{fig:PCI-in_lt_2030} to \ref{fig:PCI-in_lt_2050} compared to \textit{CP} pipeline utilisation in Figures \ref{fig:CP_lt_2030} to \ref{fig:CP_lt_2050}). Further, this reduces the reliance on larger investments into wind generation and more expensive Direct Air Capture (DAC) technologies near the sequestration sites. These effects are slightly less pronounced in the 2050 model results, system costs can be reduced by 26 to 41 bn. \euro{} p.a. with PCI-PMI and endogenous pipeline investments. 

\begin{figure*}[htbp]
  \centering
  \includegraphics[width=\textwidth]{balances_overview_co2 stored}
  \caption{\ce{CO2} balances in long-term scenarios.}
  \label{fig:balances_overview_co2_stored}
\end{figure*}

\paragraph{Carbon capture, utilisation, and storage}
\label{sec:ccus}
We find that most of the differences in system cost and savings can be attributed to the production and utilisation of \ce{CO2}, as shown in Figure \ref{fig:balances_overview_co2_stored}. Lacking the option to transport \ce{CO2} from industry and other point sources to the offshore sequestration sites, the model has to invest in expensive DAC technologies in the \textit{DI} scenario. While the sequestration target of 50 Mt p.a. in 2030 is binding for the \textit{DI} scenario, all other scenarios sequester more \ce{CO2}, the higher their \ce{CO2} pipeline build-out. The 53.9 Mt p.a. \ce{CO2} sequestered in the \textit{CP} serve as an indicator for what would be a cost-optimal amount for 2030 with perfectly located pipelines. With the inclusion of PCI-PMI projects, \ce{CO2} sequestration ranges from 58.7 Mt p.a. in the \textit{PCI} to 75 Mt p.a. in the \textit{PCI-in} scenario. 
Looking at 2040 and 2050, in place of expensive DAC in the \textit{DI} scenario, the model equips biomass-based industrial processes primarily located in Belgium, the Netherlands and Western regions of Germany (see Figures \ref{fig:PCI_lt_2030_co2}, \ref{fig:PCI_lt_2040_co2}, and \ref{fig:PCI_lt_2050_co2}). 

In 2040 and 2050, all sequestration targets (Table \ref{tab:targets}) are overachieved, as the full combined \ce{CO2} sequestration potential of 398 Mt p.a. is exploited in all scenarios where PCI-PMI projects are included (\textit{PCI} to \textit{CP}. Emissions are captured from industrial processes equipped with carbon capture units, with biomass-based industry providing the largest share in carbon capture from point sources, ranging from 119 to 241 Mt p.a. in 2040 and 149 to 287 Mt p.a. in 2050, increasing with the build-out of \ce{CO2} infrastructure (from left to right, see Figure \ref{fig:balances_overview_co2_stored}). Being the most expensive carbon capture option, only up to 8 Mt p.a. of \ce{CO2} is captured from SMR CC processes in the \textit{PCI} scenario in 2050. 
With a lower sequestration potential of 286 Mt p.a. in \textit{DI} scenario, more \ce{CO2} is used as a precursor for the synthesis of Fischer-Tropsch fuels instead --- 221 Mt p.a. vs. 115-127 Mt p.a. (2040) and 206 Mt p.a. vs 153-163 Mt p.a. (2050), to meet the emission reduction targets for 2040 and 2050, respectively. 
Given the fixed exogenous demand for (shipping) methanol (Figure \ref{fig:exogenous_demand}), \ce{CO2} demand for methanolisation is constant across all scenarios (39 Mt p.a. in 2030, 89 Mt p.a. in 2040, and 127 Mt p.a. in 2050). 

\clearpage
\begin{figure*}[htbp]
  \centering
  \begin{subfigure}[t]{0.4\textwidth}
      \vspace{0pt}
      \includegraphics[width=1\textwidth,trim=0cm 2.8cm 0cm 0cm, clip]{maps/pcipmi/base_s_adm___2030-balance_map_H2}
      \vspace{-0.5cm}
      \caption{\ce{H2} 2030.}
      \label{fig:PCI_lt_2030_h2}
  \end{subfigure}
  \hfill
  \begin{subfigure}[t]{0.4\textwidth}
      \vspace{0pt}
      \includegraphics[width=1\textwidth,trim=0cm 3.2cm 0cm 0cm, clip]{maps/pcipmi/base_s_adm___2030-balance_map_co2_stored} 
      \vspace{-0.5cm}
      \caption{\ce{CO2} 2030.}
      \label{fig:PCI_lt_2030_co2}
  \end{subfigure}
  \begin{subfigure}[t]{0.4\textwidth}
      \vspace{0pt}
      \includegraphics[width=1\textwidth,trim=0cm 2.8cm 0cm 0cm, clip]{maps/pcipmi/base_s_adm___2040-balance_map_H2}
      \vspace{-0.5cm}
      \caption{\ce{H2} 2040.}
      \label{fig:PCI_lt_2040_h2}
  \end{subfigure}
  \hfill
  \begin{subfigure}[t]{0.4\textwidth}
      \vspace{0pt}
      \includegraphics[width=1\textwidth,trim=0cm 3.2cm 0cm 0cm, clip]{maps/pcipmi/base_s_adm___2040-balance_map_co2_stored} 
      \vspace{-0.5cm}
      \caption{\ce{CO2} 2040.}
      \label{fig:PCI_lt_2040_co2}
  \end{subfigure}
  \begin{subfigure}[t]{0.4\textwidth}
      \vspace{0pt}
      \includegraphics[width=1\textwidth,trim=0cm 0cm 0cm 0cm, clip]{maps/pcipmi/base_s_adm___2050-balance_map_H2}
      \vspace{-0.5cm}
      \caption{\ce{H2} 2050.}
      \label{fig:PCI_lt_2050_h2}
  \end{subfigure}
  \hfill
  \begin{subfigure}[t]{0.4\textwidth}
      \vspace{0pt}
      \includegraphics[width=1\textwidth,trim=0cm 0cm 0cm 0cm, clip]{maps/pcipmi/base_s_adm___2050-balance_map_co2_stored} 
      \vspace{-0.7cm}
      \caption{\ce{CO2} 2050.}
      \label{fig:PCI_lt_2050_co2}
  \end{subfigure}
  \caption{\textit{PCI} long-term scenario --- Regional distribution of \ce{H2} and \ce{CO2} production, utilisation, storage, and transport.}
  \label{fig:PCI_lt_2050}
\end{figure*}
\clearpage

\begin{figure*}[htbp]
  \centering
  \includegraphics[width=\textwidth]{balances_overview_H2}
  \caption{\ce{H2} balances in long-term scenarios.}
  \label{fig:balances_overview_H2_stored}
\end{figure*}
\paragraph{Hydrogen production and utilisation}
\label{sec:h2_production_and_utilisation}
\ce{H2} production in the model is primarily driven by the demand in for Fischer-Tropsch fuels and methanolisation. In 2030 and 2050, the electrolytic \ce{H2} production target of 10 and 45 Mt p.a. is binding, equivalent to 333 and 1500 TWh p.a., at a lower heating value of \SI{33.33}{kWh/kg} for \ce{H2}. Only in 2040, the \ce{H2} production target of 27.5 Mt p.a. (917 TWh p.a.) is overachieved by 185-247 TWh p.a. in the \textit{PCI} to \textit{CP} scenarios. \ce{H2} production in the \textit{DI} is significantly higher, given its need for additional Fischer-Tropsch synthesis, as described in the previous section.
In 2050, Fischer-Tropsch fuels are used to satisfy the demand for kerosene in aviation and naphta for industrial processes (see Table \ref{fig:exogenous_demand}). Only about about 93 to 173 TWh p.a. of \ce{H2} is directly used in the industry. Throughout all long-term scenarios, \ce{H2} is almost exclusively produced via electrolysis. Only without any \ce{H2} pipeline infrastructure in the \textit{DI}, the model relies on steam methane reforming (SMR) to produce 71 to 102 TWh p.a. of \ce{H2} in 2040 and 2050, respectively.
Regionally, \ce{H2} production is concentrated in regions with high solar PV potential such as the Iberian and Italian Peninsula, as well as high wind infeed regions including Denmark, the Netherlands and Belgium. The produced \ce{H2} is then transported via \ce{H2} pipelines including PCI-PMI projects to carbon point sources  in central, continental Europe where it is used as a precursor for Fischer-Tropsch fuels. Onsite \ce{H2} production and consumption primarily occurs for methanolisation processes. Figures \ref{fig:PCI_lt_2030_h2}, \ref{fig:PCI_lt_2040_h2}, and \ref{fig:PCI_lt_2050_h2} provide a map of the regional distribution of \ce{H2} production, utilisation, and transport in the \textit{PCI} scenario.  Additional maps are provided in section \ref{sec:maps} of the appendix. Note that PCI-PMI projects or candidates (in \textit{CP} scenario) are plotted in dotted white lines.

\paragraph{Base scenario} Figure \ref{fig:balance_maps_base} shows the regional distribution of the \ce{H2} and \ce{CO2} value chain in the Base scenario. Note that for the specific year of 2030, a disconnect in \ce{H2} infrastructure between central and southeastern Europe can be observed, due to the delay in commissioning of the project connecting the two networks. Within the two interconnected regions, almost homogenous average marginal prices for \ce{H2} can be observed. Note that Figure \ref{fig:balance_map_h2_base} shows the cost of all \ce{H2} produced, weighted by the respective regional demand at a certain point in time. \ce{CO2} prices are higher in demand regions for industry processes and methanolisation located in northwestern Europe --- primarily Norway and the United Kingdom (Figure \ref{fig:balance_map_co2_base}). Negative \ce{CO2} prices in souhtheastern Europe indicate a lack of demand and missing economic value.

\subsection{Short-term scenarios}


\begin{itemize}
  \item Regarding DAC Figure \ref{fig:delta_balances_dac}
  \item No DAC in 2030 yet, primarily from CC from point sources
  \item 2040 sees strong effect in shor-term runs, delaying the pipelines means a much higher utilisation in DAC to compensate for missing pipelines
\end{itemize}

\begin{figure}[htbp]
  \centering
  \includegraphics[width=\textwidth]{delta_balances_DAC}
  \caption{Delta balances --- \ce{CO2} from Direct Air Capture.}
  \label{fig:delta_balances_dac}
\end{figure}

\begin{figure}[htbp]
  \centering
  \includegraphics[width=\textwidth]{delta_balances_H2 Electrolysis}
  \caption{Delta balances --- Electrolytic \ce{H2} production}
  \label{fig:delta_balances_h2_electrolysis}
\end{figure}



\paragraph{Scenario A compared to Base} PCI-PMI infrastructure account for a total of around 30 bn. \euro{} p.a. in additional total system costs, indicating that for the target year 2030, the projects are not cost-optimal. With a delay of PCI-PMI projects in scenario \textit{A}, Europe's policy targets can still be achieved at significantly lower cost. However, this comes at the expense of a less interconnected energy system, which may lead to higher costs in the long run. Further, \ce{H2} prices vary more strongly across regions, seeing higher costs in southeastern Europe due to industrial demand and lower renewable potentials (Figure \ref{fig:balance_map_h2_scenario_a}). We make similar observations for \ce{CO2} --- a lack of pipeline infrastructure increases spread of \ce{CO2} prices, seeing higher values for \ce{CO2} in regions with high demand (e.g. for industrial processes or methanolisation). 

\paragraph{Scenario B compared to Base}
By omitting a green \ce{H2} target, almost no electrolysers are installed. Around \SI{8}{Mt} are still produced to cover industrial \ce{H2} and methanol (primarily shipping) demand (Figures \ref{fig:h2_balance} and \ref{fig:co2_balance}). However, this demand is met by decentral steam methane reforming instead of electrolysers (Figure \ref{fig:h2_balance}). 
Without specifying a \ce{CO2} sequestration target, the system still collects around \SI{21}{Mt} of \ce{CO2} p.a. primarily from process emissions in the industry sector and sequesters it in carbon sinks near industrial sites where a sequestration potential is identified (see Figure \ref{fig:regional_scope_map}) \cite{hofmannH2CO2Network2025}. This carbon sequestration is incentivised by the emission constraint for 2030. As no pipeline infrastructure is built in these scenarios, the chosen locations differ in the delay scenarios --- this can be observed for regions near the coast, such as the United Kingdom and Norway (see Figure \ref{fig:regional_scope_map}). Given the lack of infrastructure, both the average cost for \ce{H2} and \ce{CO2} are higher in scenario \textit{B} compared to the Base scenario (Figures \ref{fig:balance_map_h2_scenario_b} and \ref{fig:balance_map_co2_scenario_b}.

Overall, the results for the modelling year 2030 show that reaching the EU's 2030 \ce{H2} production and \ce{CO2} sequestration targets translates into around 20 bn. \euro{} p.a. in total system costs for all included sectors (Figure \ref{fig:system_costs}). This is true for both comparing scenario \textit{A} and \textit{Base} scenario with scenario \textit{B}, respectively, deducting the cost of the PCI-PMI projects.

\begin{figure*}[htbp]
  \centering
  \includegraphics[width=\textwidth]{totex_heatmap.pdf}
  \caption{Annual system costs by long-term scenario and planning horizon.}
  \label{fig:totex_heatmap}
\end{figure*}

\begin{figure*}[htbp]
  \centering
  \includegraphics[width=\textwidth]{regret_matrix}
  \caption{Regret matrix. Calculating regret terms by subtracting system costs of long-term scenarios (columns) from short-term scenarios (rows). Positive values reflect higher costs in the short-term scenarios compared to the long-term ones.}
  \label{fig:regret_matrix}
\end{figure*}


\subsection{Limitations of our study}
\begin{itemize}
  \item Haversine distance for level playing field
  \item No discetisation of pipelines
  \item Regional resolution for computational reasons
  \item ...
\end{itemize}

Our study focuses primarily on the effects on real, planned infrastructure in the European energy system. Most final energy demand is given exogenously, naturally a key driver of infrastructure utilisation. We somewhat reduce the impact with the reduced targets scenario where at least the key carriers H2 and CO2 are freely optimised.

Single weather year assessment, this particular year has the properties, ...

\newpage
\section{Conclusion}
\label{sec:conclusion}
We conclude that although all three EU policy targets for 2030 can be achieved without PCI-PMI infrastructure, they bring additional benefits: i) \ce{H2} pipelines projects help distribute more affordable green H$_2$ from northern and south-western Europe to high-demand regions in central Europe; ii) \ce{CO2} transport and storage projects help decarbonising the industry by connecting major industrial sites and their process emissions to offshore sequestration sites in the North Sea (Denmark, Norway, and the Netherlands). Preliminary results have further shown that most PCI-PMI projects seem to be over-dimensioned and are not cost-optimal, as very few projects show utilisation above 1000 full-load hours. However, to adequately assess the value of PCI-PMI projects, we need to assess their benefits in future target years. Further, policy targets for 2030 are not cost-effective, although needed in the long run to reach net-zero emissions by 2050.

\paragraph{Research outlook} Next steps include the implementation of remaining PCI-PMI projects, such as hybrid offshore interconnectors (energy islands), electricity storages, and \ce{CO2} shipping routes. To evaluate the long-term value of PCI-PMI projects in a sector-coupled European energy system, we will model pathway dependencies towards 2050. We will also assess the sensitivity of the infrastructure to technology-specific build-out rates.


\newpage
\section*{CRediT authorship contribution statement}
\textbf{Bobby Xiong}: Conceptualisation, Methodology, Software, Validation, Investigation, Data Curation, Writing --- Original Draft, Review \& Editing, Visualisation. \textbf{Iegor Riepin}: Conceptualisation, Methodology, Investigation, Writing --- Review \& Editing, Project Administration, Funding acquisition. \textbf{Tom Brown}: Investigation, Resources, Writing --- Review \& Editing, Supervision, Funding acquisition.

\section*{Declaration of competing interest}
The authors declare that they have no known competing financial interests or personal relationships that could have appeared to influence the work reported in this paper.

\section*{Data and code availability}
All results, including solved PyPSA networks and summaries in .csv format are published on Zenodo: \newline
\href{https://doi.org/XX.YYYY/zenodo.10000000}{https://doi.org/XX.YYYY/zenodo.10000000}

The entire workflow, including the custom model based on PyPSA-Eur v2025.01.0, PCI-PMI project implementation, regret-matrix setup, postprocessing and visualisation routines can be completely reproduced from the GitHub repository: \newline 
\href{https://github.com/bobbyxng/pcipmi-policy-targets}{https://github.com/bobbyxng/pcipmi-policy-targets}

\section*{Acknowledgements}
This work was supported by the German Federal Ministry for Economic Affairs and Climate Action (BMWK) under Grant No. 03EI4083A (RESILIENT). This project has been funded by partners of the CETPartnership (\href{https://cetpartnership.eu}{https://cetpartnership.eu}) through the Joint Call 2022. As such, this project has received funding from the European Union's Horizon Europe research and innovation programme under grant agreement no. 101069750.


%% The Appendices part is started with the command \appendix;
%% appendix sections are then done as normal sections
\newpage
\appendix

\section{Supplementary material --- Data}
\label{app:data}\begin{figure*}[htbp]
  \centering
  \includegraphics[width=\textwidth]{exogenous_demand.pdf}
  \caption{Exogenous demand.}
  \label{fig:exogenous_demand}
\end{figure*}

\clearpage
\section{Supplementary material --- Methodology}
\label{app:methodology}

\begin{table*}[htbp]
  \centering
  \caption{Regional clustering: A total of 99 regions are modelled, excl. offshore buses.}
  \label{tab:regional_clustering}
  \scriptsize
  \begin{tabularx}{\textwidth}{R{3.9cm}>{\centering\arraybackslash}X>{\centering\arraybackslash}X}
    \toprule
     & \textbf{Country} & \textbf{Buses} \\
    \midrule
    \textbf{Administrative level} & $\bm\sum$ & \textbf{99} \\
    NUTS2 & Finland (FI) & 4 \\
          & Norway (NO) & 6 \\
    \midrule
    NUTS1 & Belgium (BE)** & 2 \\
          & Switzerland (CH) & 1 \\
          & Czech Republic (CZ) & 1 \\
          & Germany (DE)* & 13 \\
          & Denmark (DK) & 1 \\
          & Estonia (EE) & 1 \\
          & Spain (ES)* & 5 \\
          & France (FR) & 13 \\
          & Great Britain (GB)* & 11 \\
          & Greece (GR) & 3 \\
          & Ireland (IE) & 1 \\
          & Italy (IT)* & 6 \\
          & Lithuania (LT) & 1 \\
          & Luxembourg (LU) & 1 \\
          & Latvia (LV) & 1 \\
          & Montenegro (ME) & 1 \\
          & Macedonia (MK) & 1 \\
          & Netherlands (NL) & 4 \\
          & Poland (PL) & 7 \\
          & Portugal (PT) & 1 \\
          & Sweden (SE) & 3 \\
          & Slovenia (SI) & 1 \\
          & Slovakia (SK) & 1 \\
    \midrule
    NUTS0 & Albania (AL) & 1 \\
          & Austria (AT) & 1 \\
          & Bosnia and Herzegovina (BA) & 1 \\
          & Bulgaria (BG) & 1 \\
          & Croatia (HR) & 1 \\
          & Hungary (HU) & 1 \\
          & Romania (RO) & 1 \\
          & Serbia (RS) & 1 \\
          & Kosovo (XK) & 1 \\
    \bottomrule
  \end{tabularx}
  \caption*{\scriptsize City-states (*) (i.e., Berlin, Bremen, Hamburg, Madrid, and London) and regions without substations (**) (one in BE) are merged with neighbours. Sardinia and Sicily are modelled as two separate regions.}
\end{table*}

\begin{table*}
  \centering
  \caption{DUMMY Overview of technology cost assumptions. TODO compare with FLEITER PAPER TABLE 9}
  \label{tab:cost_assumptions}
  \scriptsize
  \begin{tabularx}{\textwidth}{R{3.9cm}>{\centering\arraybackslash}X>{\centering\arraybackslash}X>{\centering\arraybackslash}X>{\centering\arraybackslash}X}
    \toprule
    & \textbf{Unit} & \textbf{2030} & \textbf{2040} & \textbf{2050} \\
    \midrule
    \textbf{Technology} & & & & \\
    CO2 pipelines & XX & 1000 & 1000 & 1000 \\
    Onshore, offshore & XX & 1000 & 1000 & 1000 \\
    Electrolysers & XX & 1000 & 1000 & 1000 \\
    \bottomrule
  \end{tabularx}
\end{table*}

\clearpage
\section{Supplementary material --- Results and discussion}
\label{app:results_and_discussion}
\subsection{Installed capacities}
\label{sec:installed_capacities}
\begin{figure*}[htbp]
  \centering
  \includegraphics[width=\textwidth]{capacities_overview.pdf}
  \caption{Installed capacities in long-term scenarios.}
  \label{fig:capacities_overview}
\end{figure*}

\clearpage
\subsection{Delta capacities}
\label{sec:delta_system_costs}
\begin{figure*}[htbp]
  \centering
  \includegraphics[width=\textwidth]{capacities_overview_extended.pdf}
  \caption{$\Delta$Capacities --- Short-term minus long-term runs.}
  \label{fig:capacities_overview_extended}
\end{figure*}


\clearpage
\subsection{Delta system costs}
\label{sec:delta_system_costs}
\begin{figure*}[htbp]
  \centering
  \includegraphics[width=\textwidth]{costs_overview_extended.pdf}
  \caption{$\Delta$System costs --- Short-term minus long-term runs.}
  \label{fig:costs_overview_extended}
\end{figure*}

\clearpage
\subsection{Delta balances}
\begin{figure*}[htbp]
  \centering
  \includegraphics[width=\textwidth]{balances_overview_extended_co2 stored}
  \caption{$\Delta$\ce{CO2} balances --- Short-term minus long-term runs.}
  \label{fig:balances_overview_extended_co2_stored}
\end{figure*}
\begin{figure}[htbp]
  \centering
  \includegraphics[width=\textwidth]{delta_balances_process emissions CC}
  \caption{$\Delta$\ce{CO2} balances --- Process emissions including Carbon Capture.}
  \label{fig:delta_balances_process_emissions_CC}
\end{figure}

\begin{figure}[htbp]
  \centering
  \includegraphics[width=\textwidth]{delta_balances_solid biomass for industry CC}
  \caption{$\Delta$\ce{CO2} balances --- Carbon capture from solid biomass for industry point sources.}
  \label{fig:delta_balances_biomass_industry_cc}
\end{figure}

\begin{figure}[htbp]
  \centering
  \includegraphics[width=\textwidth]{delta_balances_SMR CC}
  \caption{$\Delta$\ce{CO2} balances --- Carbon capture from steam methane reforming point sources.}
  \label{fig:delta_balances_smr_cc}
\end{figure}

\begin{figure}[htbp]
  \centering
  \includegraphics[width=\textwidth]{delta_balances_gas for industry CC}
  \caption{$\Delta$\ce{CO2} balances --- Carbon captured from gas for industry point sources.}
  \label{fig:delta_balances_gas_for_industry}
\end{figure}

\begin{figure*}[htbp]
  \centering
  \includegraphics[width=\textwidth]{balances_overview_extended_H2}
  \caption{$\Delta$\ce{H2} balances --- Short-term minus long-term runs.}
  \label{fig:balances_overview_extended_H2_stored}
\end{figure*}

\clearpage
\subsection{Maps}
\label{sec:maps}
\subsubsection{Decentral Islands}
\begin{figure*}[htbp]
  \centering
  \begin{subfigure}[t]{0.49\textwidth}
      \vspace{0pt}
      \includegraphics[width=1\textwidth]{maps/no-pipelines-no-pcipmi/base_s_adm___2030-balance_map_H2}
      \vspace{-0.5cm}
      \caption{\ce{H2} regional balances and flows.}
      \label{fig:DI_lt_2030_h2}
  \end{subfigure}
  \hfill
  \begin{subfigure}[t]{0.49\textwidth}
      \vspace{0pt}
      \includegraphics[width=1\textwidth]{maps/no-pipelines-no-pcipmi/base_s_adm___2030-balance_map_co2_stored} 
      \vspace{-0.7cm}
      \caption{\ce{CO2} regional balances and flows.}
      \label{fig:DI_lt_2030_co2}
  \end{subfigure}
  \caption{\textit{Decentral Islands} long-term scenario (2030) --- Regional distribution of \ce{H2} and \ce{CO2} production, utilisation, storage, and transport.}
  \label{fig:DI_lt_2030}
\end{figure*}

\begin{figure*}[htbp]
  \centering
  \begin{subfigure}[t]{0.49\textwidth}
      \vspace{0pt}
      \includegraphics[width=1\textwidth]{maps/no-pipelines-no-pcipmi/base_s_adm___2040-balance_map_H2}
      \vspace{-0.5cm}
      \caption{\ce{H2} regional balances and flows.}
      \label{fig:DI_lt_2040_h2}
  \end{subfigure}
  \hfill
  \begin{subfigure}[t]{0.49\textwidth}
      \vspace{0pt}
      \includegraphics[width=1\textwidth]{maps/no-pipelines-no-pcipmi/base_s_adm___2040-balance_map_co2_stored} 
      \vspace{-0.7cm}
      \caption{\ce{CO2} regional balances and flows.}
      \label{fig:DI_lt_2040_co2}
  \end{subfigure}
  \caption{\textit{Decentral Islands} long-term scenario (2040) --- Regional distribution of \ce{H2} and \ce{CO2} production, utilisation, storage, and transport.}
  \label{fig:DI_lt_2040}
\end{figure*}

\begin{figure*}[htbp]
  \centering
  \begin{subfigure}[t]{0.49\textwidth}
      \vspace{0pt}
      \includegraphics[width=1\textwidth]{maps/no-pipelines-no-pcipmi/base_s_adm___2050-balance_map_H2}
      \vspace{-0.5cm}
      \caption{\ce{H2} regional balances and flows.}
      \label{fig:DI_lt_2050_h2}
  \end{subfigure}
  \hfill
  \begin{subfigure}[t]{0.49\textwidth}
      \vspace{0pt}
      \includegraphics[width=1\textwidth]{maps/no-pipelines-no-pcipmi/base_s_adm___2050-balance_map_co2_stored} 
      \vspace{-0.7cm}
      \caption{\ce{CO2} regional balances and flows.}
      \label{fig:DI_lt_2050_co2}
  \end{subfigure}
  \caption{\textit{Decentral Islands} long-term scenario (2050) --- Regional distribution of \ce{H2} and \ce{CO2} production, utilisation, storage, and transport.}
  \label{fig:DI_lt_2050}
\end{figure*}

\clearpage

\subsection{PCI international}

\begin{figure*}[htbp]
  \centering
  \begin{subfigure}[t]{0.49\textwidth}
      \vspace{0pt}
      \includegraphics[width=1\textwidth]{maps/pcipmi-national-international-expansion/base_s_adm___2030-balance_map_H2}
      \vspace{-0.5cm}
      \caption{\ce{H2} regional balances and flows.}
      \label{fig:PCI-in_lt_2030_h2}
  \end{subfigure}
  \hfill
  \begin{subfigure}[t]{0.49\textwidth}
      \vspace{0pt}
      \includegraphics[width=1\textwidth]{maps/pcipmi-national-international-expansion/base_s_adm___2030-balance_map_co2_stored} 
      \vspace{-0.7cm}
      \caption{\ce{CO2} regional balances and flows.}
      \label{fig:PCI-in_lt_2030_co2}
  \end{subfigure}
  \caption{\textit{PCI} long-term scenario (2030) --- Regional distribution of \ce{H2} and \ce{CO2} production, utilisation, storage, and transport.}
  \label{fig:PCI-in_lt_2030}
\end{figure*}

\begin{figure*}[htbp]
  \centering
  \begin{subfigure}[t]{0.49\textwidth}
      \vspace{0pt}
      \includegraphics[width=1\textwidth]{maps/pcipmi-national-international-expansion/base_s_adm___2040-balance_map_H2}
      \vspace{-0.5cm}
      \caption{\ce{H2} regional balances and flows.}
      \label{fig:PCI-in_lt_2040_h2}
  \end{subfigure}
  \hfill
  \begin{subfigure}[t]{0.49\textwidth}
      \vspace{0pt}
      \includegraphics[width=1\textwidth]{maps/pcipmi-national-international-expansion/base_s_adm___2040-balance_map_co2_stored} 
      \vspace{-0.7cm}
      \caption{\ce{CO2} regional balances and flows.}
      \label{fig:PCI-in_lt_2040_co2}
  \end{subfigure}
  \caption{\textit{PCI-in} long-term scenario (2040) --- Regional distribution of \ce{H2} and \ce{CO2} production, utilisation, storage, and transport.}
  \label{fig:PCI-in_lt_2040}
\end{figure*}

\begin{figure*}[htbp]
  \centering
  \begin{subfigure}[t]{0.49\textwidth}
      \vspace{0pt}
      \includegraphics[width=1\textwidth]{maps/pcipmi-national-international-expansion/base_s_adm___2050-balance_map_H2}
      \vspace{-0.5cm}
      \caption{\ce{H2} regional balances and flows.}
      \label{fig:PCI-in_lt_2050_h2}
  \end{subfigure}
  \hfill
  \begin{subfigure}[t]{0.49\textwidth}
      \vspace{0pt}
      \includegraphics[width=1\textwidth]{maps/pcipmi-national-international-expansion/base_s_adm___2050-balance_map_co2_stored} 
      \vspace{-0.7cm}
      \caption{\ce{CO2} regional balances and flows.}
      \label{fig:PCI-in_lt_2050_co2}
  \end{subfigure}
  \caption{\textit{PCI-in} long-term scenario (2050) --- Regional distribution of \ce{H2} and \ce{CO2} production, utilisation, storage, and transport.}
  \label{fig:PCI-in_lt_2050}
\end{figure*}

\clearpage

\subsubsection{Central Planning}
\begin{figure*}[htbp]
  \centering
  \begin{subfigure}[t]{0.49\textwidth}
      \vspace{0pt}
      \includegraphics[width=1\textwidth]{maps/greenfield-pipelines/base_s_adm___2030-balance_map_H2}
      \vspace{-0.5cm}
      \caption{\ce{H2} regional balances and flows.}
      \label{fig:CP_lt_2030_h2}
  \end{subfigure}
  \hfill
  \begin{subfigure}[t]{0.49\textwidth}
      \vspace{0pt}
      \includegraphics[width=1\textwidth]{maps/greenfield-pipelines/base_s_adm___2030-balance_map_co2_stored} 
      \vspace{-0.7cm}
      \caption{\ce{CO2} regional balances and flows.}
      \label{fig:CP_lt_2030_co2}
  \end{subfigure}
  \caption{\textit{Central Planning} long-term scenario (2030) --- Regional distribution of \ce{H2} and \ce{CO2} production, utilisation, storage, and transport.}
  \label{fig:CP_lt_2030}
\end{figure*}

\begin{figure*}[htbp]
  \centering
  \begin{subfigure}[t]{0.49\textwidth}
      \vspace{0pt}
      \includegraphics[width=1\textwidth]{maps/greenfield-pipelines/base_s_adm___2040-balance_map_H2}
      \vspace{-0.5cm}
      \caption{\ce{H2} regional balances and flows.}
      \label{fig:CP_lt_2040_h2}
  \end{subfigure}
  \hfill
  \begin{subfigure}[t]{0.49\textwidth}
      \vspace{0pt}
      \includegraphics[width=1\textwidth]{maps/greenfield-pipelines/base_s_adm___2040-balance_map_co2_stored} 
      \vspace{-0.7cm}
      \caption{\ce{CO2} regional balances and flows.}
      \label{fig:CP_lt_2040_co2}
  \end{subfigure}
  \caption{\textit{Central Planning} long-term scenario (2040) --- Regional distribution of \ce{H2} and \ce{CO2} production, utilisation, storage, and transport.}
  \label{fig:CP_lt_2040}
\end{figure*}

\begin{figure*}[htbp]
  \centering
  \begin{subfigure}[t]{0.49\textwidth}
      \vspace{0pt}
      \includegraphics[width=1\textwidth]{maps/greenfield-pipelines/base_s_adm___2050-balance_map_H2}
      \vspace{-0.5cm}
      \caption{\ce{H2} regional balances and flows.}
      \label{fig:CP_lt_2050_h2}
  \end{subfigure}
  \hfill
  \begin{subfigure}[t]{0.49\textwidth}
      \vspace{0pt}
      \includegraphics[width=1\textwidth]{maps/greenfield-pipelines/base_s_adm___2050-balance_map_co2_stored} 
      \vspace{-0.7cm}
      \caption{\ce{CO2} regional balances and flows.}
      \label{fig:CP_lt_2050_co2}
  \end{subfigure}
  \caption{\textit{Central Planning} long-term scenario (2050) --- Regional distribution of \ce{H2} and \ce{CO2} production, utilisation, storage, and transport.}
  \label{fig:CP_lt_2050}
\end{figure*}

\clearpage
\bibliographystyle{elsarticle-num} 
\bibliography{references.bib}

\end{document}

\endinput
%%
%% End of file `elsarticle-template-num.tex'.


%% Template related

% \section{Example Section}
% \label{sec1}
% %% Labels are used to cross-reference an item using \ref command.

% Section text. See Subsection \ref{subsec1}.

% %% Use \subsection commands to start a subsection.
% \subsection{Example Subsection}
% \label{subsec1}

% Subsection text.

% %% Use \subsubsection, \paragraph, \subparagraph commands to 
% %% start 3rd, 4th and 5th level sections.
% %% Refer following link for more details.
% %% https://en.wikibooks.org/wiki/LaTeX/Document_Structure#Sectioning_commands

% \subsubsection{Mathematics}
% %% Inline mathematics is tagged between $ symbols.
% This is an example for the symbol $\alpha$ tagged as inline mathematics.

% %% Displayed equations can be tagged using various environments. 
% %% Single line equations can be tagged using the equation environment.
% \begin{equation}
% f(x) = (x+a)(x+b)
% \end{equation}

% %% Unnumbered equations are tagged using starred versions of the environment.
% %% amsmath package needs to be loaded for the starred version of equation environment.
% \begin{equation*}
% f(x) = (x+a)(x+b)
% \end{equation*}

% %% align or eqnarray environments can be used for multi line equations.
% %% & is used to mark alignment points in equations.
% %% \\ is used to end a row in a multiline equation.
% \begin{align}
%  f(x) &= (x+a)(x+b) \\
%       &= x^2 + (a+b)x + ab
% \end{align}

% \begin{eqnarray}
%  f(x) &=& (x+a)(x+b) \nonumber\\ %% If equation numbering is not needed for a row use \nonumber.
%       &=& x^2 + (a+b)x + ab
% \end{eqnarray}

% %% Unnumbered versions of align and eqnarray
% \begin{align*}
%  f(x) &= (x+a)(x+b) \\
%       &= x^2 + (a+b)x + ab
% \end{align*}

% \begin{eqnarray*}
%  f(x)&=& (x+a)(x+b) \\
%      &=& x^2 + (a+b)x + ab
% \end{eqnarray*}

% %% Refer following link for more details.
% %% https://en.wikibooks.org/wiki/LaTeX/Mathematics
% %% https://en.wikibooks.org/wiki/LaTeX/Advanced_Mathematics

% %% Use a table environment to create tables.
% %% Refer following link for more details.
% %% https://en.wikibooks.org/wiki/LaTeX/Tables
% \begin{table}[t]%% placement specifier
% %% Use tabular environment to tag the tabular data.
% %% https://en.wikibooks.org/wiki/LaTeX/Tables#The_tabular_environment
% \centering%% For centre alignment of tabular.
% \begin{tabular}{l c r}%% Table column specifiers
% %% Tabular cells are separated by &
%   1 & 2 & 3 \\ %% A tabular row ends with \\
%   4 & 5 & 6 \\
%   7 & 8 & 9 \\
% \end{tabular}
% %% Use \caption command for table caption and label.
% \caption{Table Caption}\label{fig1}
% \end{table}


% %% Use figure environment to create figures
% %% Refer following link for more details.
% %% https://en.wikibooks.org/wiki/LaTeX/Floats,_Figures_and_Captions
% \begin{figure}[t]%% placement specifier
% %% Use \includegraphics command to insert graphic files. Place graphics files in 
% %% working directory.
% \centering%% For centre alignment of image.
% \includegraphics{example-image-a}
% %% Use \caption command for figure caption and label.
% \caption{Figure Caption}\label{fig1}
% %% https://en.wikibooks.org/wiki/LaTeX/Importing_Graphics#Importing_external_graphics
% \end{figure}

%% For citations use: 
%%       \cite{<label>} ==> [1]

%%
% Example citation, See \cite{lamport94}.

%% If you have bib database file and want bibtex to generate the
%% bibitems, please use

%% else use the following coding to input the bibitems directly in the
%% TeX file.

%% Refer following link for more details about bibliography and citations.
%% https://en.wikibooks.org/wiki/LaTeX/Bibliography_Management

% \begin{thebibliography}{00}

% %% For numbered reference style
% %% \bibitem{label}
% %% Text of bibliographic item

% \bibitem{lamport94}
%   Leslie Lamport,
%   \textit{\LaTeX: a document preparation system},
%   Addison Wesley, Massachusetts,
%   2nd edition,
%   1994.

% \end{thebibliography}