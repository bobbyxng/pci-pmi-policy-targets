\documentclass[10pt]{article}
\usepackage[left=2.7cm,right=2.7cm,top=2.5cm,bottom=2.5cm]{geometry}
% -------------------------------------------------------------------
\usepackage[english]{babel}
\usepackage[utf8]{inputenc}									
\usepackage[T1]{fontenc}										% 
% -------------------------------------------------------------------
\usepackage{amsmath,amsfonts,amssymb,amsthm,cancel,siunitx,
calculator,calc,mathtools,empheq,latexsym}
% -------------------------------------------------------------------
\usepackage{subfig,epsfig,tikz,float}
\usepackage{hyperref}
\usepackage{xcolor}
\hypersetup{
    colorlinks,
    linkcolor={black!50!black},
    citecolor={blue!50!blue},
    urlcolor={blue!80!blue}
}
% -------------------------------------------------------------------
\usepackage{booktabs,multicol,multirow,tabularx,array}
\usepackage{lipsum}
\usepackage{csquotes}
\usepackage[super]{nth}
% -------------------------------------------------------------------
% \setlength{\parindent}{0pt}
\setlength{\parskip}{5pt}
% \textheight 19.5cm
\columnsep .5cm
% -------------------------------------------------------------------
\title{\renewcommand{\baselinestretch}{1.17}\normalsize\bf%
\uppercase{Title tbd.}
}
% -------------------------------------------------------------------
% Autorias
\author{%
Bobby Xiong$^{1,*}$, Iegor Riepin$^{1}$, Tom Brown$^{1}$\\ 
}
% -------------------------------------------------------------------
\begin{document}

\date{}

\maketitle

\vspace{-0.5cm}

\begin{center}
{\footnotesize 
$^1$Technische Universität Berlin, Department of Digital Transformation in Energy Systems, Germany \\
$^*$Presenting author: \href{mailto:xiong@tu-berlin.de}{xiong@tu-berlin.de}
}\\
\bigskip
\footnotesize
\textbf{Tags}: resilience, infrastructure, energy system modelling, energy policy \\
\bigskip
\textbf{\nth{43} International Energy Workshop --- relevant conference topics:}\\(1) Reaching net-zero emissions and climate neutrality \textbullet{} (2) Role of renewable energy in the energy transition \textbullet{} (3) Role of hydrogen, ammonia, e-fuels and e-methane in the energy transition \textbullet{} (4) Managing power system transitions --- integration of variable renewable energy and power-to-X \textbullet{} (5) Sectoral pathways for the energy transition --- transport, industry, and buildings \textbullet{} (6) Energy transition infrastructure --- assessment of infrastructure to enable the energy transition, including electrical transmission, storage, EV charging, and hydrogen distribution, CCS and CDR \textbullet{} (12) Climate resilience of energy systems \textbullet{} (13) Utilisation of scenarios by governments
\end{center}

% -------------------------------------------------------------------

\section*{Summary}
% 100-200 words

Summary

\section*{Introduction}

In the research project RESILIENT\footnote{\href{https://resilient-project.github.io/}{https://resilient-project.github.io/}}, our team develops the first truly multi-vector energy infrastructure planning tool that can handle uncertain environments. We build upon the open-source, widely-used, multi-vector energy planning tool PyPSA-Eur\footnote{\href{https://pypsa-eur.readthedocs.io/en/latest/}{https://pypsa-eur.readthedocs.io/}}, and improve its ability to optimise energy infrastructure in a resilient way. 

\section*{Methodology}

Methodology

\section*{Results (preliminary)}

Results

\section*{Conclusion}

Conclusion

\section*{References}

References

% % -----------------------------
% \bibliographystyle{IEEEtran}
% \bibliography{refs.bib}

\end{document}