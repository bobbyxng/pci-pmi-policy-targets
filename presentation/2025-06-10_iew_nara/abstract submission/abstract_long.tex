%% 
%% Copyright 2007-2025 Elsevier Ltd
%% 
%% This file is part of the 'Elsarticle Bundle'.
%% ---------------------------------------------
%% 
%% It may be distributed under the conditions of the LaTeX Project Public
%% License, either version 1.3 of this license or (at your option) any
%% later version.  The latest version of this license is in
%%    http://www.latex-project.org/lppl.txt
%% and version 1.3 or later is part of all distributions of LaTeX
%% version 1999/12/01 or later.
%% 
%% The list of all files belonging to the 'Elsarticle Bundle' is
%% given in the file `manifest.txt'.
%% 
%% Template article for Elsevier's document class `elsarticle'
%% with numbered style bibliographic references
%% SP 2008/03/01
%% $Id: elsarticle-template-num.tex 272 2025-01-09 17:36:26Z rishi $
%%
% \documentclass[preprint,12pt]{elsarticle}

%% Use the option review to obtain double line spacing
%% \documentclass[authoryear,preprint,review,12pt]{elsarticle}

%% Use the options 1p,twocolumn; 3p; 3p,twocolumn; 5p; or 5p,twocolumn
%% for a journal layout:
%% \documentclass[final,1p,times]{elsarticle}
% \documentclass[final,1p,times,twocolumn]{elsarticle}
%% \documentclass[final,3p,times]{elsarticle}
% \documentclass[final,3p,times,twocolumn]{elsarticle}
%% \documentclass[final,5p,times]{elsarticle}
\documentclass[final,5p,times,twocolumn]{elsarticle}

%% For including figures, graphicx.sty has been loaded in
%% elsarticle.cls. If you prefer to use the old commands
%% please give \usepackage{epsfig}

%% The amssymb package provides various useful mathematical symbols
\usepackage{amssymb}
%% The amsmath package provides various useful equation environments.
\usepackage{amsmath}
%% The amsthm package provides extended theorem environments
%% \usepackage{amsthm}

%% The lineno packages adds line numbers. Start line numbering with
%% \begin{linenumbers}, end it with \end{linenumbers}. Or switch it on
%% for the whole article with \linenumbers.
%% \usepackage{lineno}

%% Additional packages
\usepackage[super]{nth}
\usepackage{amsfonts,amsthm,cancel,siunitx,
calculator,calc,mathtools,empheq,latexsym}
\usepackage[version=4]{mhchem}
\usepackage[breaklinks]{hyperref}
\usepackage{booktabs,multicol,multirow,tabularx,array}
\usepackage[official]{eurosym}
\usepackage{subcaption}

% Remove preprint submitted to...
\makeatletter
\def\ps@pprintTitle{
  \let\@oddhead\@empty
  \let\@evenhead\@empty
  \def\@oddfoot{\reset@font\hfil}
  \def\@evenfoot{\reset@font\hfil}
}
\makeatother

% Aliases
\let\autocite\cite

\graphicspath{
    {figures/}
  }
\DeclareGraphicsExtensions{.pdf,.jpeg,.jpg,.png}


\begin{document}

\begin{frontmatter}

%% Title, authors and addresses

%% use the tnoteref command within \title for footnotes;
%% use the tnotetext command for theassociated footnote;
%% use the fnref command within \author or \affiliation for footnotes;
%% use the fntext command for theassociated footnote;
%% use the corref command within \author for corresponding author footnotes;
%% use the cortext command for theassociated footnote;
%% use the ead command for the email address,
%% and the form \ead[url] for the home page:
\title{The Role of Projects of Common Interest in Reaching Europe's Energy Policy Targets\tnoteref{label1}}
\tnotetext[label1]{working title}
\author[affil1]{Bobby Xiong\corref{cor1}}
\cortext[cor1]{Presenting author (\href{mailto:xiong@tu-berlin.de}{xiong@tu-berlin.de})}
\affiliation[affil1]{organization={Technische Universität Berlin, Department of Digital Transformation in Energy Systems},
            % addressline={},
            city={Berlin},
            % postcode={},
            % state={},
            country={Germany}}

\author[affil1]{Iegor Riepin}
\author[affil1]{Tom Brown}

% \title{}

%% use optional labels to link authors explicitly to addresses:
%% \author[label1,label2]{}
%% \affiliation[label1]{organization={},
%%             addressline={},
%%             city={},
%%             postcode={},
%%             state={},
%%             country={}}
%%
%% \affiliation[label2]{organization={},
%%             addressline={},
%%             city={},
%%             postcode={},
%%             state={},
%%             country={}}


%% Abstract
\begin{abstract}
  The European Union (EU) aims to achieve climate-neutrality by 2050 --- with ambitious 2030 targets --- including a \SI{55}{\percent} reduction in greenhouse gas emissions, \SI{10}{Mt} p.a. of green \ce{H2} production, and \SI{50}{Mt} p.a. of \ce{CO2} sequestration. The European Commission bi-annually selects so-called Projects of Common Interest (PCI) and Projects of Mutual Interest (PMI) which are of transnational importance as they link the energy systems of European countries. 
  Using the open-source, sector-coupled energy system model PyPSA-Eur, we assess the impact of PCI-PMI projects on the EU energy system, focusing on power, heat, transport, industry, and agriculture. We look into how a delay of such projects may impact reaching the EU's policy targets, and explore how the different policy targets conflict with the overall greenhouse gas reduction target. While preliminary results for 2030 suggest that the policy targets can be achieved even without PCI-PMI projects, they bring additional benefits: i) \ce{H2} pipelines improve affordability and distribution of green \ce{H2} and kickstart the hydrogen economy, ii) \ce{CO2} transport projects connect major industrial emissions to offshore sequestration sites in the North Sea.
  Next steps include incorporating all remaining PCI-PMI projects, i.e. hybrid interconnectors and \ce{CO2} shipping routes, as well as the assessment of long-term pathway effects towards 2050. Our findings underscore the interplay between cross-border cooperation, infrastructure investments, and policy targets in the European energy transition throughout all sectors.
\end{abstract}

%%Graphical abstract
% \begin{graphicalabstract}
% %\includegraphics{grabs}
% \end{graphicalabstract}

%%Research highlights
% \begin{highlights}
% \item Research highlight 1
% \item Research highlight 2
% \end{highlights}

%% Keywords
\begin{keyword}
energy system modelling \sep energy policy \sep infrastructure \sep resilience 
%% keywords here, in the form: keyword \sep keyword

%% PACS codes here, in the form: \PACS code \sep code

%% MSC codes here, in the form: \MSC code \sep code
%% or \MSC[2008] code \sep code (2000 is the default)

\end{keyword}

\end{frontmatter}

%% Add \usepackage{lineno} before \begin{document} and uncomment 
%% following line to enable line numbers
%% \linenumbers

%% main text
%%

\section*{\nth{43} IEW relevant conference topics }
Reaching net-zero emissions and climate neutrality (1) Role of renewable energy in the energy transition (2) Role of hydrogen, ammonia, e-fuels and e-methane in the energy transition (3) Managing power system transitions --- integration of variable renewable energy and power-to-X (4) Sectoral pathways for the energy transition --- transport, industry, and buildings (5) Energy transition infrastructure --- assessment of infrastructure to enable the energy transition, including electrical transmission, storage, EV charging, and hydrogen distribution, CCS and CDR (6) Climate resilience of energy systems (12) Utilisation of scenarios by governments (13)

\section{Introduction and motivation}
\label{sec:introduction}

On the pathway to a climate-neutral Europe by 2050, the European Union (EU) has set ambitious targets for 2030, three of the most prominent including a reduction of \SI{55}{\percent} in greenhouse gas emissions \autocite{europeancommissionFit55Delivering2021}, \SI{10}{Mt} p.a. green \ce{H2} production \autocite{europeancommissionREPowerEUPlanCommunication2022}, and \SI{50}{Mt} p.a. \ce{CO2} sequestration \autocite{europeanparliamentRegulationEU20242024}.

To support reaching these targets, the EU has identified a list of Projects of Common Interest (PCI) which are key cross-border infrastructure projects that link the energy systems of EU countries, ranging from storages, over transmission lines to pipelines and carbon sinks \autocite{europeancommissionCommissionDelegatedRegulation2023}, based on projects submitted by transmission system operators, consortia, or third parties. Projects of Mutual Interest (PMI) further include cooperations with countries outside the EU, such as Norway or the United Kingdom. With a PCI-PMI status, project awardees receive strong political support and are, amongst others, eligible for financial support (e.g. through funding of the Connecting Europe Facility) and see accelerated permitting processes. On the other hand, project promoters are obliged to undergo comprehensive reporting and monitoring processes. In order for projects to be eligible for PCI-PMI status, their potential benefits need to outweigh their costs \autocite{europeancommissionCommissionDelegatedRegulation2023}. Given the political and lighthouse character, these projects are highly likely to be implemented. However, any large infrastructure project, including PCI-PMI projects, commonly face delays due to permitting, financing, procurement bottlenecks, etc. \autocite{acerConsolidatedReportProgress2023}.

\paragraph{Research questions} From the ambitious policy targets and the nature of PCI-PMI projects, we derive the following key research questions:

\begin{enumerate}
    \item What is the cost of sticking to EU policy targets?
    \item Do the hydrogen and sequestration targets conflict with meeting the greenhouse gas target cost-effectively?
 \end{enumerate}

\section{Methodology}
\label{sec:methodology}

We use the open-source, sector-coupled energy system model PyPSA-Eur \cite{neumannPotentialRoleHydrogen2023,frysztackiComparisonClusteringMethods2022,glaumOffshorePowerHydrogen2024,horschPyPSAEurOpenOptimisation2018} to optimise investment into generation, storage, and transmission infrastructure (including electricity, natural gas, hydrogen, \ce{CO2}, and Power-to-X conversion) as well as operation/dispatch. The model is spatially and temporally highly resolved and covers the entire European continent, including stocks of existing power plants \autocite{gotzensPerformingEnergyModelling2019}, renewable potentials, and availability time series \autocite{hofmannAtliteLightweightPython2021}. It covers today's high-voltage transmission grid (AC \SI{220}{kV} to \SI{750}{kV} and DC \SI{150}{kV} upwards) \autocite{xiongModellingHighVoltageGrid2024}.

\subsection{Feature implementation}
\label{sec:feature_implementation}

By accessing the REST API\footnote{Representational State Transfer Application Programming Interface} of the PCI-PMI Transparency Platform \autocite{europeancommissionPCIPMITransparencyPlatform2024} and associated public project sheets provided by the European Commission, we implement the PCI-PMI projects into the PyPSA-Eur model to assess their impact in the power, heat, transport, industry, feedstock, and agriculture sector. Note that we use standardised costs for all PCI-PMI projects \autocite{zeyenPyPSATechnologydataV0922024} for two reasons: i) Cost data provided by project promoters can be incomplete and may not include the same cost components, and ii) to ensure comparability as well as level-playing field between all potential projects, including both PCI-PMI and model-endogenous investments.
Our implementation can adapt to the needs and configuration of the model, including selected technologies, geographical and temporal resolution, as well as the level of sector-coupling. An overview of the implemented PCI-PMI projects is shown in Figure \ref{fig:pci_pmi_projects_map}.

\begin{figure}[t]
  \centering
  \includegraphics[width=\linewidth]{pci_pmi_projects_map}
  \caption{PCI-PMI projects implemented in the PyPSA-Eur model as of the date of submission. Own illustration based on data from the European Commission \autocite{europeancommissionPCIPMITransparencyPlatform2024}.}
  \label{fig:pci_pmi_projects_map}
\end{figure}


\subsection{Scenario setup}
\label{sec:scenario_setup}

As of the date of submission, we model three key scenarios for the target year 2030 which will set the base year for pathways towards 2050: a \textit{Base} scenario in which policy targets are achieved and all projects are commissioned on time as well as two PCI-PMI delay scenarios \textit{A} and \textit{B}. Table \ref{tab:scenarios} gives an overview of the scenarios' key assumptions and their differences. Depending on the scenario, we formulate and activate additional constraints to ensure the fulfilment of the EU policy targets.
\begin{table}[t]
  \centering
  \renewcommand{\arraystretch}{1.1}
  \scriptsize % Reduces the font size
  \caption{Initial scenario setup. Own illustration.}
  \begin{tabular}{lccc}
      \toprule
      Scenario & Base & A. All targets & B. Emission target \\
      \midrule
      PCI-PMI projects & on time & delayed & delayed \\
      \midrule
      \ce{CO2} emission & \SI{-55}{\percent}/\SI{2}{Gt} p.a. & \SI{-55}{\percent}/\SI{2}{Gt} p.a. & \SI{-55}{\percent}/\SI{2}{Gt} p.a. \\
      \ce{CO2} sequestration & \SI{50}{Mt} p.a. & \SI{50}{Mt} p.a. & --- \\
      Green \ce{H2} & \SI{10}{Mt} p.a. & \SI{10}{Mt} p.a. & --- \\
      \midrule
      \ce{CO2} seq. sites & PCI-PMI & endogeneous & endogeneous \\
      \ce{H2} storage & PCI-PMI & endogeneous & endogeneous \\
      \midrule
      \ce{CO2} pipelines & \parbox[t]{2cm}{\centering PCI-PMI and \\ endog. expansion} & --- & --- \\
      \ce{H2} pipelines & \parbox[t]{2cm}{\centering PCI-PMI and \\ endog. expansion} & --- & --- \\
      AC/DC lines & PCI-PMI & --- & --- \\
      \bottomrule
  \end{tabular}
  \label{tab:scenarios}
\end{table}
We solve all scenarios by minimising total system costs, resolving 34 countries to 90 buses at 3-hourly temporal resolution.

\begin{figure}[!b]
  \centering
  \includegraphics[width=\linewidth]{system_costs}
  \caption{Results --- Total system costs by technology and infrastructure. Own illustration.}
  \label{fig:system_costs}
\end{figure}

\section{Results --- preliminary}
\label{sec:results}

First results for the modelling year 2030 show that reaching the EU's 2030 \ce{H2} production and \ce{CO2} sequestration targets translates into around 20 bn. \euro{} p.a. in total system costs for all included sectors (Figure \ref{fig:system_costs}). This is true for both comparing scenario \textit{A} and \textit{Base} scenario with scenario \textit{B}, respectively, deducting the cost of the PCI-PMI projects. Assuming standardised cost assumptions for all PCI-PMI projects. 

\begin{figure*}[!htbp]
  \centering
  \begin{subfigure}[t]{0.47\textwidth} % [t] aligns at the top
      \vspace{0pt}
      \includegraphics[width=\textwidth]{balance_map_h2_base} % Replace with your image
      \vspace{0.3cm}
      \caption{\ce{H2} regional balances and flows (all \ce{H2} produced).}
      \label{fig:balance_map_h2_base}
  \end{subfigure}
  \hfill
  \begin{subfigure}[t]{0.47\textwidth} % [t] aligns at the top
      \vspace{0pt}
      \includegraphics[width=\textwidth]{balance_map_co2_base} % Replace with your image
      \caption{\ce{CO2} regional balances and flows.}
      \label{fig:balance_map_co2_base}
  \end{subfigure}
  \caption{Results \textit{Base} scenario --- Regional distribution of \ce{H2} and \ce{CO2} production, utilisation, storage, and transport in the Base scenario. Own illustration.}
  \label{fig:balance_maps_base}
\end{figure*}

\paragraph{Base scenario} Figure \ref{fig:balance_maps_base} shows the regional distribution of the \ce{H2} and \ce{CO2} value chain in the Base scenario. Note that for the specific year of 2030, a disconnect in \ce{H2} infrastructure between central and southeastern Europe can be observed, due to the delay in commissioning of the project connecting the two networks. Within the two interconnected regions, almost homogenous average marginal prices for \ce{H2} can be observed. Note that Figure \ref{fig:balance_map_h2_base} shows the cost of all \ce{H2} produced, weighted by the respective regional demand at a certain point in time. \ce{CO2} prices are higher in demand regions for industry processes and methanolisation located in northwestern Europe --- primarily Norway and the United Kingdom (Figure \ref{fig:balance_map_co2_base}). Negative \ce{CO2} prices in souhtheastern Europe indicate a lack of demand and missing economic value.
Utilisation of \ce{H2} pipelines vary strongly across the PCI-PMI projects. In most of the times, pipelines serve the purpose of transporting \ce{H2} in a single direction only, i.e. from high renewable potential regions to \ce{H2} consumption sites, where it serves as a precursor for methanolisation or direct use in industry and shipping (see Figure \ref{fig:balance_map_h2_base}). Prominent PCI-PMI projects with particularly high full-load hours include P9.9.2 \textit{Hydrogen Interconnector Denmark-Germany} (\SI{6937}{h}) and P11.2  \textit{Nordic-Baltic Hydrogen Corridor} (\SI{2295}{h}), followed by projects connecting major steel-industrial and chemical sites of Germany (southwest) with Belgium (P9.4 \textit{H2ercules West}, \SI{1634}{h}), the Netherlands (P9.6 \textit{Netherlands National Hydrogen Backbone}, \SI{1967}{h} and P9.7.3 \textit{Delta Rhine Corridor H2}, \SI{1510}{h}), and France (P9.2.2 \textit{MosaHYyc}, \SI{4662}{h}). PCI project P13.8 \textit{EU2NSEA} connects \ce{CO2} from process emissions in Germany, Belgium and the Netherlands to major geological sequestration sinks close to the Norwegian shore \textit{Smeaheia} and \textit{Luna} with an annual injection potential of \SI{20}{Mt} p.a. and {5}{Mt} p.a., respectively. 


\paragraph{Scenario A compared to Base} PCI-PMI infrastructure account for a total of around 30 bn. \euro{} p.a. in additional total system costs, indicating that for the target year 2030, the projects are not cost-optimal. With a delay of PCI-PMI projects in scenario \textit{A}, Europe's policy targets can still be achieved at significantly lower cost. However, this comes at the expense of a less interconnected energy system, which may lead to higher costs in the long run. Further, \ce{H2} prices vary more strongly across regions, seeing higher costs in southeastern Europe due to industrial demand and lower renewable potentials (Figure \ref{fig:balance_map_h2_scenario_a}). We make similar observations for \ce{CO2} --- a lack of pipeline infrastructure increases spread of \ce{CO2} prices, seeing higher values for \ce{CO2} in regions with high demand (e.g. for industrial processes or methanolisation). 

\paragraph{Scenario B compared to Base}
By omitting a green \ce{H2} target, almost no electrolysers are installed. Around \SI{8}{Mt} are still produced to cover industrial \ce{H2} and methanol (primarily shipping) demand (Figures \ref{fig:h2_balance} and \ref{fig:co2_balance}). However, this demand is met by decentral steam methane reforming instead of electrolysers (Figure \ref{fig:h2_balance}). 
Without specifying a \ce{CO2} sequestration target, the system still collects around \SI{21}{Mt} of \ce{CO2} p.a. primarily from process emissions in the industry sector and sequesters it in carbon sinks near industrial sites where a sequestration potential is identified (see Figure \ref{fig:sequestration_map}) \autocite{hofmannH2CO2Network2024}. This carbon sequestration is incentivised by the emission constraint for 2030. As no pipeline infrastructure is built in these scenarios, the chosen locations differ in the delay scenarios --- this can be observed for regions near the coast, such as the United Kingdom and Norway (see Figure \ref{fig:sequestration_map}). Given the lack of infrastructure, both the average cost for \ce{H2} and \ce{CO2} are higher in scenario \textit{B} compared to the Base scenario (Figures \ref{fig:balance_map_h2_scenario_b} and \ref{fig:balance_map_co2_scenario_b}.


\section{Conclusion --- preliminary}
\label{sec:conclusion}
We conclude that while all three EU policy targets for 2030 can be achieved without PCI-PMI infrastructure, they bring additional benefits: i) \ce{H2} pipelines projects help distribute more affordable green H$_2$ from northern and south-western Europe to high-demand regions in central Europe; ii) \ce{CO2} transport and storage projects help decarbonising the industry by connecting major industrial sites and their process emissions to offshore sequestration sites in the North Sea (Denmark, Norway, and the Netherlands). Preliminary results have further shown that most PCI-PMI projects seem to be over-dimensioned and are not cost-optimal, as very few projects show full-load hours of above \SI{1000}{} p.a. However, to adequately assess the value of PCI-PMI projects, we need to assess their benefits in future target years. Further, policy targets for 2030 are not cost-effective, although needed in the long run to reach net-zero emissions by 2050.

\paragraph{Research outlook} Next steps include the implementation of remaining PCI-PMI projects, such as hybrid offshore interconnectors (energy islands), electricity storages, and \ce{CO2} shipping routes. To evaluate the long-term value of PCI-PMI projects in a sector-coupled European energy system, we will model pathway dependencies towards 2050. We will also assess the sensitivity of the infrastructure to technology-specific build-out rates.

\section*{Acknowledgements}
This work was supported by the German Federal Ministry for Economic Affairs and Climate Action (BMWK) under Grant No. 03EI4083A (RESILIENT). This project has been funded by partners of the CETPartnership (\href{https://cetpartnership.eu}{https://cetpartnership.eu}) through the Joint Call 2022. As such, this project has received funding from the European Union's Horizon Europe research and innovation programme under grant agreement no. 101069750.

\appendix

\section{Additional input data}
\label{sec:app_additional}

\begin{figure}[h]
  \centering
  \includegraphics[width=\linewidth]{sequestration_map}
  \caption{Regional sequestration potentials in scenarios \textit{A} and \textit{B} according to \autocite{hofmannH2CO2Network2024}.}
  \label{fig:sequestration_map}
\end{figure}

\newpage
\section{Additional results}
\label{sec:app_hydrogen_and_carbon_balance}

\begin{figure}[h]
  \centering
  \includegraphics[width=\linewidth]{h2_balance}
  \caption{Results --- \ce{H2} balance. Own illustration.}
  \label{fig:h2_balance}
\end{figure}

\begin{figure}[h]
  \centering
  \includegraphics[width=\linewidth]{co2_balance}
  \caption{Results --- \ce{CO2} balance. Own illustration.}
  \label{fig:co2_balance}
\end{figure}

\begin{figure*}[!htbp]
  \centering
  \begin{subfigure}[t]{0.47\textwidth}
      \vspace{0pt}
      \includegraphics[width=\textwidth]{balance_map_h2_scenario_a}
      \vspace{0.3cm}
      \vspace{-0.3cm}
      \caption{\ce{H2} regional balances and flows (Scenario A, all \ce{H2} produced).}
      \label{fig:balance_map_h2_scenario_a}
  \end{subfigure}
  \hfill
  \begin{subfigure}[t]{0.47\textwidth}
      \vspace{0pt}
      \includegraphics[width=\textwidth]{balance_map_co2_scenario_a}
      \vspace{-0.7cm}
      \caption{\ce{CO2} regional balances and flows (Scenario A).}
      \label{fig:balance_map_co2_scenario_a}
  \end{subfigure}

  \vspace{0.2cm} % Adds spacing between rows

  \begin{subfigure}[t]{0.47\textwidth}
      \vspace{0pt}
      \includegraphics[width=\textwidth]{balance_map_h2_scenario_b}
      \vspace{0.3cm}
      \vspace{-0.3cm}
      \caption{\ce{H2} regional balances and flows (Scenario B, all \ce{H2} produced).}
      \label{fig:balance_map_h2_scenario_b}
  \end{subfigure}
  \hfill
  \begin{subfigure}[t]{0.47\textwidth}
      \vspace{0pt}
      \includegraphics[width=\textwidth]{balance_map_co2_scenario_b}
      \vspace{-0.7cm}
      \caption{\ce{CO2} regional balances and flows (Scenario B).}
      \label{fig:balance_map_co2_scenario_b}
  \end{subfigure}

  \caption{Results scenarios \textit{A} and \textit{B} --- Regional distribution of \ce{H2} and \ce{CO2} production, utilisation, storage, and transport. Own illustration.}
  \label{fig:balance_maps_scenarios_a_b}
\end{figure*}


\newpage

\bibliographystyle{elsarticle-num} 
\bibliography{references.bib}

\end{document}

\endinput